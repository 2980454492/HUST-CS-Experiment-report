\documentclass[supercite]{Experimental_Report}

\title{~~~~~~C语言程序设计~~~~~~}
\author{贾柠泽}
\school{计算机科学与技术学院}
\classnum{CS2306}
\stunum{U202315594}
\date{\today}

\usepackage{algorithm, multirow}
\usepackage{algpseudocode}
\usepackage{amsmath}
\usepackage{amsthm}
\usepackage{framed}
\usepackage{mathtools}
\usepackage{subcaption}
\usepackage{xltxtra} %提供了针对XeTeX的改进并且加入了XeTeX的LOGO, 自动调用xunicode宏包(提供Unicode字符宏)
\usepackage{bm}
\usepackage{tikz}
\usepackage{tikzscale}
\usepackage{pgfplots}
\usepackage{graphicx}
\usepackage{wrapfig}
%\usepackage{enumerate}

\pgfplotsset{compat=1.16}

\geometry{left=2.5cm,bottom=2cm,top=3cm,right=2.5cm}

\newcommand{\cfig}[3]{
  \begin{figure}[htb]
    \centering
    \includegraphics[width=#2\textwidth]{images/#1.tikz}
    \caption{#3}
    \label{fig:#1}
  \end{figure}
}

\newcommand{\sfig}[3]{
  \begin{subfigure}[b]{#2\textwidth}
    \includegraphics[width=\textwidth]{images/#1.tikz}
    \caption{#3}
    \label{fig:#1}
  \end{subfigure}
}

\newcommand{\xfig}[3]{
  \begin{figure}[htb]
    \centering
    #3
    \caption{#2}
    \label{fig:#1}
  \end{figure}
}

\newcommand{\rfig}[1]{\autoref{fig:#1}}
\newcommand{\ralg}[1]{\autoref{alg:#1}}
\newcommand{\rthm}[1]{\autoref{thm:#1}}
\newcommand{\rlem}[1]{\autoref{lem:#1}}
\newcommand{\reqn}[1]{\autoref{eqn:#1}}
\newcommand{\rtbl}[1]{\autoref{tbl:#1}}

\algnewcommand\Null{\textsc{null }}
\algnewcommand\algorithmicinput{\textbf{Input:}}
\algnewcommand\Input{\item[\algorithmicinput]}
\algnewcommand\algorithmicoutput{\textbf{Output:}}
\algnewcommand\Output{\item[\algorithmicoutput]}
\algnewcommand\algorithmicbreak{\textbf{break}}
\algnewcommand\Break{\algorithmicbreak}
\algnewcommand\algorithmiccontinue{\textbf{continue}}
\algnewcommand\Continue{\algorithmiccontinue}
\algnewcommand{\LeftCom}[1]{\State $\triangleright$ #1}

\newtheorem{thm}{定理}[section]
\newtheorem{lem}{引理}[section]

\colorlet{shadecolor}{black!15}

\theoremstyle{definition}
\newtheorem{alg}{算法}[section]

\def\thmautorefname~#1\null{定理~#1~\null}
\def\lemautorefname~#1\null{引理~#1~\null}
\def\algautorefname~#1\null{算法~#1~\null}

\begin{document}

\maketitle

\begin{abstract}

线性代数是数学的一个分支,它的研究对象是向量,向量空间(或称线性空间),线性变换和有限维的线性方程组。向量空间是现代数学的一个重要课题;因而,线性代数被广泛地应用于抽象代数和泛函分析中;通过解析几何,线性代数得以被具体表示。线性代数的理论已被泛化为算子理论。由于科学研究中的非线性模型通常可以被近似为线性模型,使得线性代数被广泛地应用于自然科学和社会科学中。

线性代数是在数学以及物理领域的一种强大的算法工具,其中有特别多让人脑洞大开的结论和眼前一亮的公式,该实验报告将以我现有的知识和局限的视角带领大家领略线性代数的奥秘和其中的底层逻辑。

\end{abstract}

\clearpage

\pagenumbering{Roman}

\tableofcontents[level=2]
\clearpage

\pagenumbering{arabic}

\section{行列式}

行列式的原始含义就是经过变换之后图形面积变化的倍数。行列式是我们学习线性代数的一个核心知识之一。行列式的本质是一个数,即一个矩阵或一组数字经过行列式计算后的结果就是一个数字,该数字的几何含义就是一个几何图形经过该行列式矩阵的变换之后所改改变的面积。

\subsection{二阶行列式求解}

\left | \begin{matrix}
a &b \\
c &d \\
\end{matrix} \right | 
=ad-bc.

若设矩阵A有\[
A=\begin{bmatrix}
a & b \\
c & d \\ 
\end{bmatrix}
\]
则行列式记作D,且D=ad-bc.

\subsection{三阶行列式求解}

\left | \begin{matrix}
a_1_1&a_1_2&a_1_3\\
a_2_1&a_2_2&a_2_3\\
a_3_1&a_3_2&a_3_3\\
\end{matrix} \right | 
=a_1_1a_2_2a_3_3+a_1_2a_2_3a_3_1+a_1_3a_2_1a_3_2-a_1_3a_2_2a_3_1-a_1_2a_2_1a_3_3-a_1_1a_2_3a_3_2.

\subsection{n阶行列式求解}
记行列式D=
\left | \begin{matrix}
a_1_1&a_1_2&...&a_1_n\\
a_2_1&a_2_2&...&a_2_n\\
...&...&&...\\
a_n_1&a_n_2&...&a_n_n\\
\end{matrix} \right | 

若D的阶数大于3,则通常情况下二阶和三阶的办法将会非常复杂,计算D的方法是通过行列式的行变换或列变换将D等价形成一个上三角行列式或者下三角行列式,在之后会讲到行列式的性质一章中会有解题办法。

\subsection{行列式的性质}

行列式具有以下性质:

1.行列式的交换性质:交换矩阵的两行(列)会改变行列式的符号。

\left | \begin{matrix}
a_1_1&a_1_2&...&a_1_n\\
a_2_1&a_2_2&...&a_2_n\\
...&...&&...\\
a_n_1&a_n_2&...&a_n_n\\
\end{matrix} \right | =-\left | \begin{matrix}
a_2_1&a_2_2&...&a_2_n\\
a_1_1&a_1_2&...&a_1_n\\
...&...&&...\\
a_n_1&a_n_2&...&a_n_n\\
\end{matrix} \right | 

2.行列式与转置矩阵:矩阵的行列式等于其转置矩阵的行列式。

\left | \begin{matrix}
a_1_1&a_1_2&...&a_1_n\\
a_2_1&a_2_2&...&a_2_n\\
...&...&&...\\
a_n_1&a_n_2&...&a_n_n\\
\end{matrix} \right | =\left | \begin{matrix}
a_1_1&a_2_1&...&a_n_1\\
a_1_2&a_2_2&...&a_n_2\\
...&...&&...\\
a_1_n&a_2_n&...&a_n_n\\
\end{matrix} \right | 

3.行列式的线性性质:如果将矩阵的某一行(列)乘以一个数然后加到另一行(列)上,行列式不变。

\left | \begin{matrix}
ka_1_1&a_1_2&...&a_1_n\\
ka_2_1&a_2_2&...&a_2_n\\
...&...&&...\\
ka_n_1&a_n_2&...&a_n_n\\
\end{matrix} \right | =k\left | \begin{matrix}
a_1_1&a_1_2&...&a_1_n\\
a_2_1&a_2_2&...&a_2_n\\
...&...&&...\\
a_n_1&a_n_2&...&a_n_n\\
\end{matrix} \right | 

4.行列式的可加性:如果矩阵的两行(列)相等,则行列式为0。

\left | \begin{matrix}
a_1_1&a_1_2&...&a_1_n\\
a_2_1&a_2_2&...&a_2_n\\
...&...&&...\\
a_n_1&a_n_2&...&a_n_n\\
\end{matrix} \right | =\left | \begin{matrix}
a_1_1&a_1_1&...&a_1_n\\
a_2_1&a_2_1&...&a_2_n\\
...&...&&...\\
a_n_1&a_n_1&...&a_n_n\\
\end{matrix} \right | =0

5.行列式的乘法性质:如果矩阵A可逆,则其行列式|A|不为0。

6.用矩阵的一行(列)加上另一行(列)的倍数,行列式不变。

\left | \begin{matrix}
a_1_1&a_1_2&...&a_1_n\\
a_2_1&a_2_2&...&a_2_n\\
...&...&&...\\
a_n_1&a_n_2&...&a_n_n\\
\end{matrix} \right | =\left | \begin{matrix}
a_1_1&ka_1_1+a_1_2&...&a_1_n\\
a_2_1&ka_2_1+a_2_2&...&a_2_n\\
...&...&&...\\
a_n_1&ka_n_1+a_n_2&...&a_n_n\\
\end{matrix} \right | 

7.当矩阵的某一行(列)全为零的时候,行列式为零。

\left | \begin{matrix}
a_1_1&a_1_2&...&a_1_n\\
0&0&...&0\\
...&...&&...\\
a_n_1&a_n_2&...&a_n_n\\
\end{matrix} \right | =0

8.如果矩阵是三角形的,那么行列式等于对角线上元素的乘积。

\left | \begin{matrix}
a_1_1&a_1_2&...&a_1_n\\
0&a_2_2&...&a_2_n\\
...&...&&...\\
0&0&...&a_n_n\\
\end{matrix} \right |=a_1_1a_2_2...a_n_n

\subsection{行列式按行展开}

行列式按行展开是一种计算行列式的方法,它是通过将行列式按照其中的某一行进行展开,然后再对展开的结果进行计算得出行列式的值。

在数学中,行列式按行展开的公式可以表示为:
$|D|=a_1_1A_1_1+a_1_2A_1_2+...+a_1_nA_1_n.$其中$a_i_j$表示矩阵 D 中第 i 行第 j 列的元素,$A_i_j$表示代数余子式。

例如,
A_1_2=\left | \begin{matrix}
a_2_1&a_2_3&...&a_2_n\\
a_3_1&a_3_3&...&a_3_n\\
...&...&&...\\
a_n_1&a_n_3&...&a_n_n\\
\end{matrix} \right |

\subsection{克莱姆法则}

在解线性方程组时,除了可以使用高斯消元法,还可以使用克莱姆法则这种方法。

\subsubsection{非齐次线性方程组}
设有一个非齐次线性方程组:

\begin{equation}
\begin{cases}
a_1_1x_1+a_1_2x_2+...+a_1_nx_n=b_1\\
a_2_1x_1+a_2_2x_2+...+a_2_nx_n=b_2\\
...\\
a_n_1x_1+a_n_2x_2+...+a_n_nx_n=b_n
\end{cases}
\end{equation}

分别求解以下行列式

D=\left | \begin{matrix}
a_1_1&a_1_2&...&a_1_n\\
a_2_1&a_2_2&...&a_2_n\\
...&...&&...\\
a_n_1&a_n_2&...&a_n_n\\
\end{matrix} \right |

D_1=\left | \begin{matrix}
b_1&a_1_2&...&a_1_n\\
b_2&a_2_2&...&a_2_n\\
...&...&&...\\
b_n&a_n_2&...&a_n_n\\
\end{matrix} \right |

D_2=\left | \begin{matrix}
a_1_1&b_1&...&a_1_n\\
a_2_1&b_2&...&a_2_n\\
...&...&&...\\
a_n_1&b_n&...&a_n_n\\
\end{matrix} \right |


...


D_n=\left | \begin{matrix}
a_1_1&a_1_2&...&b_1\\
a_2_1&a_2_2&...&b_2\\
...&...&&...\\
a_n_1&a_n_2&...&b_n\\
\end{matrix} \right |

所以求解得方程组的解x_i=\frac{D_i}{D}.

\subsubsection{齐次线性方程组}

设有一个齐次线性方程组:

\begin{equation}
\begin{cases}
a_1_1x_1+a_1_2x_2+...+a_1_nx_n=0\\
a_2_1x_1+a_2_2x_2+...+a_2_nx_n=0\\
...\\
a_n_1x_1+a_n_2x_2+...+a_n_nx_n=0
\end{cases}
\end{equation}

求解行列式D同上.

若D=0,则该齐次线性方程组只有零解,即$x_i=0$,i=1,2,...n.

若$D \neq 0$,则该齐次线性方程组线性相关,存在无数组解。

\newpage

\section{矩阵}

矩阵的几何意义对应的一种变换,不同的矩阵对应着不同的矩阵,当然,特定的矩阵变换会有着一些重要的性质,要想理清楚矩阵的特殊含义,首先我们要学习矩阵的一些基本性质。

\subsection{矩阵的基本概念}

矩阵是由一组数按照特定的摆放顺序而组成的一个数表。矩阵的表达形式一般为$A_m_×_n$,其中m为矩阵的行数,n为矩阵的列数,所以
\[
A_m_×_n=\begin{bmatrix}
a_1_1&a_1_2&...&a_1_n\\
a_2_1&a_2_2&...&a_2_n\\
...&...&&...\\
a_m_1&a_m_2&...&a_m_n
\end{bmatrix}
\]

也可以表示为:
\[
A_m_×_n=\begin{pmatrix}
a_1_1&a_1_2&...&a_1_n\\
a_2_1&a_2_2&...&a_2_n\\
...&...&&...\\
a_m_1&a_m_2&...&a_m_n
\end{pmatrix}
\]

\subsection{特殊矩阵}

1.方块矩阵:简称方阵,即矩阵A的行数和列数相等,记作$A_n$。

\[
A_n=\begin{bmatrix}
a_1_1&a_1_2&...&a_1_n\\
a_2_1&a_2_2&...&a_2_n\\
...&...&&...\\
a_n_1&a_n_2&...&a_n_n
\end{bmatrix}
\]

2.单位矩阵:若一个方阵的主对角线上元素都是1,其余位置元素都是0,则该矩阵称为单位矩阵,记作$I_n$或者$E_n$。

\[
I_n=\begin{bmatrix}
1&&&\\
&1&&\\
&&...&\\
&&&1
\end{bmatrix}
\]

3.转置矩阵:对原矩阵进行一次转置运算得到转置矩阵,即矩阵的每个元素$a_i_j$换到新矩阵的位置为$a_j_i$,这样就得到了转置矩阵,转置后的矩阵记作$A^T$。

\[
A=\begin{bmatrix}
a_1_1&a_1_2&...&a_1_n\\
a_2_1&a_2_2&...&a_2_n\\
...&...&&...\\
a_m_1&a_m_2&...&a_m_n
\end{bmatrix}
\]

\[
A^T=\begin{bmatrix}
a_1_1&a_2_1&...&a_m_1\\
a_1_2&a_2_2&...&a_m_2\\
...&...&&...\\
a_1_n&a_2_n&...&a_m_n
\end{bmatrix}
\]

4.对称矩阵:一个方阵主对角线上元素无要求,其他位置上的元素满足$a_i_j$=$a_j_i$,则该矩阵为对称矩阵,记作$A=A^T$。

5.反对称矩阵:一个方阵主对角线上的元素均为0,其余位置上的元素满足$a_i_j$=-$a_j_i$,则该矩阵为反对称矩阵。

6.数量矩阵:主对角线上的元素相同且不为0,其余位置上的元素均为0,该矩阵被称为数量矩阵,记作A=aE。

\[
A_n=\begin{bmatrix}
a&&&\\
&a&&\\
&&...&\\
&&&a
\end{bmatrix}
\]

7.对角形矩阵:只有主对角线上的元素不全为0,其余位置的元素均为0的矩阵叫做对角形矩阵,记作$diag(a_1,a_2,...,a_n)$。

\[
diag(a_1,a_2,...,a_n)=\begin{bmatrix}
a_1&&&\\
&a_2&&\\
&&...&\\
&&&a_n
\end{bmatrix}
\]

\subsection{矩阵的运算}

\subsubsection{加法运算}

加法运算是矩阵中比较简单的运算形式。两个矩阵能够进行加法运算的前提条件是这两个矩阵必须是同形矩阵,加法运算如下所示:

\[
\begin{bmatrix}
a_1_1&a_1_2&...&a_1_n\\
a_2_1&a_2_2&...&a_2_n\\
...&...&&...\\
a_m_1&a_m_2&...&a_m_n
\end{bmatrix}+\begin{bmatrix}
b_1_1&b_1_2&...&b_1_n\\
b_2_1&b_2_2&...&b_2_n\\
...&...&&...\\
b_m_1&b_m_2&...&b_m_n
\end{bmatrix}=\begin{bmatrix}
a_1_1+b_1_1&a_1_2+b_1_2&...&a_1_n+b_1_n\\
a_2_1+b_2_1&a_2_2+b_2_2&...&a_2_n+b_2_n\\
...&...&&...\\
a_m_1+b_m_1&a_m_2+b_m_2&...&a_m_n+b_m_n
\end{bmatrix}
\]

\subsubsection{乘法运算}

矩阵的乘法运算是矩阵运算的核心和难点。两个矩阵能够相乘的前提条件是前一个矩阵的列数等于后一个矩阵的行数,最终结果的矩阵的行数等于前一个矩阵的行数,列数等于后一个矩阵的列数,用符号语言描述则是$A_m_×_n×B_n_×_k=C_m_×_K$.

\[
\begin{bmatrix}
a_1_1&a_1_2&...&a_1_n\\
a_2_1&a_2_2&...&a_2_n\\
...&...&&...\\
a_m_1&a_m_2&...&a_m_n
\end{bmatrix}\begin{bmatrix}
b_1_1&b_1_2&...&b_1_k\\
b_2_1&b_2_2&...&b_2_k\\
...&...&&...\\
b_n_1&b_n_2&...&b_n_k
\end{bmatrix}=\begin{bmatrix}
c_1_1&c_1_2&...&c_1_k\\
c_2_1&c_2_2&...&c_2_k\\
...&...&&...\\
c_m_1&c_m_2&...&c_m_k
\end{bmatrix}
\]

其中,$c_t_h=\sum_{i=1}^{n} a_t_ib_i_h$

\subsection{方阵的行列式}

1.$|A|=|A^T|$.

2.$|kA|=k^n|A|$.

3.|AB|=|A||B|.

\subsection{伴随矩阵}

\subsubsection{伴随矩阵的定义}

矩阵A为n阶方阵,则A的伴随矩阵中的每一个元素都是A方阵中该位置元素的对称位置上的元素的代数余子式,A的伴随矩阵记作$A^*$。

\[
A^*=\begin{bmatrix}
A_1_1&A_2_1&...&A_n_1\\
A_1_2&A_2_2&...&A_n_2\\
...&...&&...\\
A_1_n&A_2_n&...&A_n_n
\end{bmatrix}
\]

\subsubsection{伴随矩阵的性质}

1.AA^*=A^*A=|A|E.

2.$|A||A^*|=|A|^n$.

3.$|A^*|=|A|^n^-^1$.

4.$(kA)^*=k^n^-^1A^*$.

\subsection{逆矩阵}

\subsubsection{逆矩阵的定义}

设矩阵A,B均为n阶方阵,记作$A_n$、$B_n$,若满足A×B=B×A=E则称B是A的逆矩阵(也可以称A是B的逆矩阵),记作$B=A^-^1$。

\subsubsection{逆矩阵的性质}

1.逆矩阵存在性:一个矩阵存在逆矩阵的前提是它是一个可逆矩阵,即行列式不为0。

2.逆矩阵的唯一性:一个矩阵的逆矩阵是唯一的。

3.逆矩阵的乘法:两个矩阵的逆矩阵相乘等于它们的乘积的逆矩阵,即$(AB)^-^1=B^-^1A^-^1$。

4.转置矩阵的逆矩阵:一个矩阵的转置矩阵的逆矩阵等于原矩阵的逆矩阵的转置,即$(A^T)^-^1=(A^-^1)^T$。

5.逆矩阵和伴随矩阵的关系:$A^*=|A|A^-^1$。

\subsubsection{逆矩阵的求解方法}

1.伴随矩阵法,$A^-^1=\frac{A^*}{|A|}$.

2.初等行变换法,将A构造成[A|E]的形式,经过初等行变换就可以变为$[E|A^-^1]$,这样就得到了A的逆矩阵。

\subsection{初等变换}

\subsubsection{初等变换的形式}

1.交换两行(列)。

例如交换方阵A的第一行和第二行,交换的过程以及交换后的结果为

\[
\begin{bmatrix}
&1&&&\\
1&&&&\\
&&1&&\\
&&&...&\\
&&&&1
\end{bmatrix}\begin{bmatrix}
a_1_1&a_1_2&...&a_1_n\\
a_2_1&a_2_2&...&a_2_n\\
a_3_1&a_3_2&...&a_3_n\\
...&...&&...\\
a_n_1&a_n_2&...&a_n_n
\end{bmatrix}=\begin{bmatrix}
a_2_1&a_2_2&...&a_2_n\\
a_1_1&a_1_2&...&a_1_n\\
a_3_1&a_3_2&...&a_3_n\\
...&...&&...\\
a_n_1&a_n_2&...&a_n_n
\end{bmatrix}
\]

2.将某一行(列)扩大到原来的k倍($k \neq 0$)。

例如将方阵A的第二行扩大k倍,交换的过程以及交换后的结果为

\[
\begin{bmatrix}
1&&&\\
&k&&\\
&&...&\\
&&&&1
\end{bmatrix}\begin{bmatrix}
a_1_1&a_1_2&...&a_1_n\\
a_2_1&a_2_2&...&a_2_n\\
...&...&&...\\
a_n_1&a_n_2&...&a_n_n
\end{bmatrix}=\begin{bmatrix}
a_1_1&a_1_2&...&a_1_n\\
ka_2_1&ka_2_2&...&ka_2_n\\
...&...&&...\\
a_n_1&a_n_2&...&a_n_n
\end{bmatrix}
\]

3.将某一行(列)的k倍加到另一行上去($k \neq 0$)。

例如将方阵A的第一行的k倍加到第二行上去,交换的过程和结果为

\[
\begin{bmatrix}
1&k&&\\
&1&&\\
&&...&\\
&&&&1
\end{bmatrix}\begin{bmatrix}
a_1_1&a_1_2&...&a_1_n\\
a_2_1&a_2_2&...&a_2_n\\
...&...&&...\\
a_n_1&a_n_2&...&a_n_n
\end{bmatrix}=\begin{bmatrix}
a_1_1&a_1_2&...&a_1_n\\
ka_1_1+a_2_1&ka_1_2+a_2_2&...&ka_1_n+a_2_n\\
...&...&&...\\
a_n_1&a_n_2&...&a_n_n
\end{bmatrix}
\]

\subsubsection{初等变换的性质}

1.一个矩阵经过初等变换,其可逆性不变。

2.若一个矩阵左乘变换矩阵,则是对该矩阵进行行变换;若一个矩阵右乘变换矩阵,则是对该矩阵进行列变换。

例如将方阵A的第二行扩大k倍,交换的过程以及交换后的结果为

\[
\begin{bmatrix}
1&&&\\
&k&&\\
&&...&\\
&&&&1
\end{bmatrix}\begin{bmatrix}
a_1_1&a_1_2&...&a_1_n\\
a_2_1&a_2_2&...&a_2_n\\
...&...&&...\\
a_n_1&a_n_2&...&a_n_n
\end{bmatrix}=\begin{bmatrix}
a_1_1&a_1_2&...&a_1_n\\
ka_2_1&ka_2_2&...&ka_2_n\\
...&...&&...\\
a_n_1&a_n_2&...&a_n_n
\end{bmatrix}
\]

若将方阵A的第二列扩大k倍,交换的过程以及交换后的结果为

\[
\begin{bmatrix}
a_1_1&a_1_2&...&a_1_n\\
a_2_1&a_2_2&...&a_2_n\\
...&...&&...\\
a_n_1&a_n_2&...&a_n_n
\end{bmatrix}\begin{bmatrix}
1&&&\\
&k&&\\
&&...&\\
&&&&1
\end{bmatrix}=\begin{bmatrix}
a_1_1&ka_1_2&...&a_1_n\\
a_2_1&ka_2_2&...&a_2_n\\
...&...&&...\\
a_n_1&ka_n_2&...&a_n_n
\end{bmatrix}
\]

3.等价矩阵:若矩阵A经过初等行变换和列变换能够得到矩阵B,则称A和B等价,可以写为PAQ=B。

4.行简化阶梯型矩阵:矩阵A经过行变换可以变为行简化阶梯型

\subsection{矩阵的秩}

\subsubsection{秩的性质}

1.秩等于矩阵的标准形中1的个数。

2.矩阵初等行变换,秩不变,即r(A)=r(PA)=r(AQ)=r(PAQ)。

3.等价矩阵的秩相等。

\subsubsection{有关秩的不等式}

1.r(A)+r(B)≤r(AB)+n.

2.r(A+B)≤r(A)+r(B).

3.r(AB)≤min\{r(A),r(B)\}.

4.r(A_m_×_n)≤min\{m,n\}.

\nocite{*} %% 作用是不对文献进行引用,但可以生成文献列表

\bibliographystyle{HustGraduPaper}
\bibliography{HustGraduPaper}

\end{document}