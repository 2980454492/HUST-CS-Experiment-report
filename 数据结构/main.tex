\documentclass[supercite]{Experimental_Report}

\title{~~~~~~C语言程序设计~~~~~~}
\author{贾柠泽}
\school{计算机科学与技术学院}
\classnum{CS2306}
\stunum{U202315594}
\instructor{周全}
\date{\today}

\usepackage{algorithm, multirow}
\usepackage{algpseudocode}
\usepackage{amsmath}
\usepackage{amsthm}
\usepackage{framed}
\usepackage{mathtools}
\usepackage{subcaption}
\usepackage{xltxtra} %提供了针对XeTeX的改进并且加入了XeTeX的LOGO, 自动调用xunicode宏包(提供Unicode字符宏)
\usepackage{bm}
\usepackage{tikz}
\usepackage{tikzscale}
\usepackage{pgfplots} %用于绘制高质量图形
\usepackage{graphicx} %用于在文档中插入和管理图片
\usepackage{wrapfig} %将文字环绕在图片或表格周围
\usepackage{listings} %可以插入源代码
%\usepackage{enumerate}

\pgfplotsset{compat=1.16}

\geometry{left=2.5cm,bottom=2cm,top=3cm,right=2.5cm}

\newcommand{\cfig}[3]{
  \begin{figure}[htb]
    \centering
    \includegraphics[width=#2\textwidth]{images/#1.tikz}
    \caption{#3}
    \label{fig:#1}
  \end{figure}
}

\newcommand{\sfig}[3]{
  \begin{subfigure}[b]{#2\textwidth}
    \includegraphics[width=\textwidth]{images/#1.tikz}
    \caption{#3}
    \label{fig:#1}
  \end{subfigure}
}

\newcommand{\xfig}[3]{
  \begin{figure}[htb]
    \centering
    #3
    \caption{#2}
    \label{fig:#1}
  \end{figure}
}

\newcommand{\rfig}[1]{\autoref{fig:#1}}
\newcommand{\ralg}[1]{\autoref{alg:#1}}
\newcommand{\rthm}[1]{\autoref{thm:#1}}
\newcommand{\rlem}[1]{\autoref{lem:#1}}
\newcommand{\reqn}[1]{\autoref{eqn:#1}}
\newcommand{\rtbl}[1]{\autoref{tbl:#1}}

\algnewcommand\Null{\textsc{null }}
\algnewcommand\algorithmicinput{\textbf{Input:}}
\algnewcommand\Input{\item[\algorithmicinput]}
\algnewcommand\algorithmicoutput{\textbf{Output:}}
\algnewcommand\Output{\item[\algorithmicoutput]}
\algnewcommand\algorithmicbreak{\textbf{break}}
\algnewcommand\Break{\algorithmicbreak}
\algnewcommand\algorithmiccontinue{\textbf{continue}}
\algnewcommand\Continue{\algorithmiccontinue}
\algnewcommand{\LeftCom}[1]{\State $\triangleright$ #1}

\newtheorem{thm}{定理}[section]
\newtheorem{lem}{引理}[section]

\colorlet{shadecolor}{black!15}

\theoremstyle{definition}
\newtheorem{alg}{算法}[section]

\def\thmautorefname~#1\null{定理~#1~\null}
\def\lemautorefname~#1\null{引理~#1~\null}
\def\algautorefname~#1\null{算法~#1~\null}

\begin{document}

\maketitle

\clearpage

\pagenumbering{Roman}

\tableofcontents[level=2]
\clearpage

\pagenumbering{arabic}

\section{基于链式存储结构的线性表实现}

\subsection{问题描述}

链式线性表具有前一实验中顺序线性表的多数特点,比如其数据元素在逻辑位置上相邻。与其不同的是,链式线性表在物理存储中并不相邻。

链式表分为单链表,循环链表和双向链表。链表可能会设置为空的头结点便于操作,也可能设置尾指针。

本实验的目的是制作一个有空头结点的单链表,并对其能够实现进行一些基本和进阶操作。
    
\subsection{系统设计}

链式线性表中数据元素的物理位置分开存储,这就要求使用指针来实现。对
应每个数据节点,需要增加一个指向下一节点的指针,即指针域,用于串联整个
线性表。因此,链式线性表的结构声明如下:

\begin{lstlisting}
typedef struct LNode { //单链表(链式结构)结点的定义
	ElemType data;
	struct LNode *next;
} LNode, *LinkList;

typedef struct LIST{ // 多链表的管理表定义
    struct { 
        char name[30];
        LinkList L;
    } elem[10];
    int length ;
    int listsize ;
}LISTS;
\end{lstlisting}

\subsection{系统实现}

\subsubsection{线性表的创建}

函数名称:InitList(\&L)

初始条件:线性表L不存在

操作结果:构造一个空的线性表

\begin{figure}[htb]
	\begin{center}
		\includegraphics[scale=1]{1-1.png}
		\caption{线性表的创建}
	\end{center}
\end{figure}

\subsubsection{线性表的销毁}

函数名称:DestroyList(\&L)

初始条件:线性表L已存在

操作结果:删除线性表

\subsubsection{清空线性表}

函数名称:ClearList(\&L)

初始条件:线性表L已存在

操作结果:将L重置为空表

\begin{figure}[htb]
	\begin{center}
		\includegraphics[scale=1]{1-2.png}
		\caption{清空线性表之后的结果}
	\end{center}
\end{figure}

\subsubsection{线性表判空}

函数名称:ListEmpty( L)

初始条件:线性表L已存在

操作结果:若L为空表,返回OK,否则返回ERROR

\begin{figure}[htb]
	\begin{center}
		\includegraphics[scale=1]{1-2.png}
		\caption{线性表为空}
	\end{center}
\end{figure}

\begin{figure}[htb]
	\begin{center}
		\includegraphics[scale=1]{1-3.png}
		\caption{线性表不为空}
	\end{center}
\end{figure}

\subsubsection{求线性表长度}

函数名称:ListLength(L)

初始条件:线性表L已存在

操作结果:返回L中数据元素的个数

\begin{figure}[htb]
	\begin{center}
		\includegraphics[scale=0.7]{1-6.png}
		\caption{求线性表长度}
	\end{center}
\end{figure}

\subsubsection{获取元素}

函数名称:GetElem(L,i,\&e)

初始条件:线性表L已存在,1<=i<=ListLength(L)

操作结果:用e返回L中的第i个元素的值

\begin{figure}[htb]
	\begin{center}
		\includegraphics[scale=0.5]{1-5.png}
		\caption{获取i位置的元素e}
	\end{center}
\end{figure}

\newpage

\subsubsection{查找元素}

函数名称:LocateElem(L,e)

初始条件:线性表L已存在

操作结果:返回e的结点位置

\begin{figure}[htb]
	\begin{center}
		\includegraphics[scale=0.7]{1-4.png}
		\caption{查找元素e}
	\end{center}
\end{figure}

\subsubsection{获取前驱元素}

函数名称:PriorElem(L,e,\&pre)

初始条件:线性表L已存在

操作结果:获取线性表L中元素e的前驱,保存在pre中,返回OK;如果没有前驱,返回ERROR

\begin{figure}[htb]
	\begin{center}
		\includegraphics[scale=0.7]{1-7.png}
		\caption{获取前驱元素}
	\end{center}
\end{figure}

\subsubsection{获取后继元素}

函数名称:NextElem(L,e,\&next)

初始条件:线性表L已存在

操作结果:获取线性表L中元素e的后继,保存在next中,返回OK;如果没有后继,返回ERROR

\begin{figure}[htb]
	\begin{center}
		\includegraphics[scale=0.7]{1-8.png}
		\caption{获取后继元素}
	\end{center}
\end{figure}

\subsubsection{插入元素}

函数名称:ListInsert(\&L,i,e)

初始条件:线性表L已存在,1<=i<=ListLength(L)+1

操作结果:在L中第i个位置之前插入新的元素数据e,L的长度+1

\begin{figure}[htb]
	\begin{center}
		\includegraphics[scale=0.5]{1-9.png}
		\caption{插入e前}
	\end{center}
\end{figure}

\begin{figure}[htb]
	\begin{center}
		\includegraphics[scale=0.5]{1-10.png}
		\caption{插入e后}
	\end{center}
\end{figure}

\subsubsection{删除元素}

函数名称:ListDelete(\&L,i,\&e)

初始条件:线性表L已存在且非空,1<=i<=ListLength(L)

操作结果:删除L中第i个数据元素,用e返回其值,L的长度-1

\begin{figure}[htb]
	\begin{center}
		\includegraphics[scale=0.5]{1-11.png}
		\caption{删除e前}
	\end{center}
\end{figure}

\begin{figure}[htb]
	\begin{center}
		\includegraphics[scale=0.5]{1-12.png}
		\caption{删除e后}
	\end{center}
\end{figure}

\subsubsection{遍历线性表}

函数名称:ListTraverse(L)

初始条件:线性表L已存在

操作结果:依次访问L中的每个元素

\subsection{系统测试}

本章使用的具有菜单的功能演示系统如图\ref{1-13}

\begin{figure}[htb]
	\begin{center}
		\includegraphics[scale=0.8]{1-13.png}
		\caption{演示系统}
        \label{1-13}
	\end{center}
\end{figure}

当用功能1第一次创建线性表时会显示创建成功。

用功能10依次插入5、9、7、6、8,我们以7为例子,输出插入成功。

用功能12遍历链表,输出5 9 7 6 8。

用功能5求一下链表长度,答案是5。

用功能7获取元素6的位置,得到位置是4。

用功能6获取第3个位置的元素,得到元素7。

用功能8获取9的前驱元素,得到元素5。

用功能9获取9的后继元素,得到元素7。

用功能11删除第4个元素,输出删除成功,并再次用功能12遍历链表。

用功能4判空,结果为不为空。

用功能3清空线性表,并求链表长度,结果为0。

用功能2销毁线性表,销毁成功。

最后输入0,退出演示系统。

\begin{figure}
	\begin{center}
            \includegraphics[scale=1]{1-14.png}
            \includegraphics[scale=1]{1-15.png}
            \includegraphics[scale=1]{1-16.png}
            \includegraphics[scale=1.2]{1-17.png}
            \includegraphics[scale=1]{1-18.png}
            \includegraphics[scale=1]{1-19.png}
        \caption{系统测试}
        \label{1-14}
	\end{center}
\end{figure}

\begin{figure}
	\begin{center}
            \includegraphics[scale=1]{1-20.png}
		\includegraphics[scale=1]{1-21.png}
            \includegraphics[scale=1]{1-22.png}	
            \includegraphics[scale=1]{1-23.png}
            \includegraphics[scale=1]{1-24.png}
            \includegraphics[scale=1]{1-25.png}
            \includegraphics[scale=1]{1-26.png}
            \includegraphics[scale=1]{1-27.png}
            \includegraphics[scale=1]{1-28.png}
        \caption{系统测试}
        \label{1-14}
	\end{center}
\end{figure}

\newpage

\subsection{实验小结}

在链表的实验中,我学到了很多与链表数据结构相关的知识,也提升了自己的编程能力和问题解决能力。以下是我对链表实验的小结:

1.理解了链表的原理和特点:通过实验,我深入理解了链表的定义、结构和操作方式,包括单链表、双向链表等不同类型的链表,以及链表的优缺点。

2.实现了链表的基本操作:在实验过程中,我编写了链表的插入、删除、查找等基本操作的代码,加深了对链表操作的理解和掌握。

3.解决了链表相关问题:在实验过程中,我遇到了一些链表相关的问题,如反转链表、检测链表是否有环等,通过实践和调试,我提升了自己对链表问题的解决能力。

4.熟悉了链表的应用场景:链表在实际项目中有广泛的应用,如实现队列、栈、LRU缓存等数据结构,通过实验,我了解了链表在实际项目中的应用场景。

5.编程技能得到提升:通过实验的编程实践,我熟练掌握了链表的实现和操作方法,提升了自己的数据结构和算法编程能力。

总的来说,通过链表实验,我不仅学到了关于链表的知识和算法,还提升了编程能力和解决问题的能力,这对我今后在计算机科学领域的学习和发展都有很大帮助。

\section{基于二叉链表的二叉树实现}

\subsection{问题描述}

二叉树(binary tree)是指树中节点的度不大于2的有序树,它是一种最简单且最重要的树。

二叉树的递归定义为:二叉树是一棵空树,或者是一棵由一个根节点和两棵互不相交的,分别称作根的左子树和右子树组成的非空树;左子树和右子树又同样都是二叉树。

二叉树(Binarytree)是树形结构的一个重要类型。许多实际问题抽象出来的数据结构往往是二叉树形式,即使是一般的树也能简单地转换为二叉树,而且二叉树的存储结构及其算法都较为简单,因此二叉树显得特别重要。二叉树特点是每个节点最多只能有两棵子树,且有左右之分。

二叉树是n个有限元素的集合,该集合或者为空、或者由一个称为根的元素及两个不相交的、被分别称为左子树和右子树的二叉树组成,是有序树。当集合为空时,称该二叉树为空二叉树。在二叉树中,一个元素也称作一个结点。

1. 结点:包含一个数据元素及若干指向子树分支的信息

2. 结点的度:一个节点拥有子树的数目称为结点的度

3. 叶子结点:也称为终端结点,没有子树的结点或者度为零的结点

4. 分支结点:也称为非终端结点,度不为零的结点称为非终端结点

5. 树的度:树中所有结点的度的最大值

6. 结点的层次:从根结点开始,假设根结点为第1层,根结点的子结点为第2层,依此类推,如果某一个结点位于第L层,则其子结点位于第L+1层

7. 树的深度:也称为树的高度,树中所有结点的层次最大值称为树的深度

8. 有序树:如果树中各棵子树的次序是有先后次序,则称该树为有序树

9. 无序树:如果树中各棵子树的次序没有先后次序,则称该树为无序树

10. 森林:由m(m≥0)棵互不相交的树构成一片森林。如果把一棵非空的树的根结点删除,则该树就变成了一片森林,森林中的树由原来根结点的各棵子树构成

\subsection{系统设计}

二叉树既可以用顺序表按层存储,也可以用二叉链表存储。本次实验中,将二叉树用二叉链表实现。将每个结点设置两个指针域,分别指向其左右孩子,即可将每个结点连接起来。最后给出根结点的指针。
\begin{lstlisting}
typedef struct {  //二叉树结点类型定义
	KeyType key;
	char others[20];
}TElemType;
typedef struct BiTNode { //二叉链表结点的定义
	TElemType data;
	struct BiTNode *lchild, *rchild;
}BiTNode,*BiTree;

typedef struct moretrees{ //多二叉树的管理表定义
    struct { 
        char name[30];
        BiTree T;
    } elem[10];
    int length ;
}BiTrees;

int front=0, rear=0;

void EnQueue(BiTree *a ,BiTree node){ //入队函数
    a[rear++]=node;
}

BiTNode* DeQueue(BiTNode **a){ //出队函数
    return a[front++];
}

void visit(BiTree T) { //二叉树访问
    printf("%d,%s ", T->data.key, T->data.others);
}
\end{lstlisting}

\subsection{系统实现}

\subsubsection{创建二叉树}

函数名称:CreateBiTree(\&T,definition[],begin) {

初始条件:二叉树已存在

操作结果:按definition构造二叉树T

实现思想:首先判断在definition中是否有相同的关键字,若出现相同直接返回ERROR;采用静态变量k防止递归过程中k的值重新定义。递归创建二叉树的思路是先创建根结点,然后递归创建左子树和右子树。当definition.key=0时,则表示该插入的结点是空结点,definition.key=-1时表示插入结束,begin的作用是判断是否是头结点。

\subsubsection{销毁二叉树}

函数名称:DestroyBiTree(\&T)

初始条件:二叉树T已存在

操作结果:销毁二叉树T

实现思想:采用递归的思想,依次遍历左右子树,先后释放孩子结点以及根结点的空间。

\subsubsection{清空二叉树}

函数名称:ClearBiTree(\&T)

初始条件:二叉树T已存在

操作结果:将二叉树T清空为空树

实现思想:将各个结点之间的父子关系删去,及将左右孩子指针指向空。

\subsubsection{二叉树判空}

函数名称:BiTreeEmpty(T)

初始条件:二叉树T已存在

操作结果:若T为空二叉树,则返回OK,否则返回ERROR

实现思想:直接判断头结点是否为空。

\subsubsection{求二叉树深度}

函数名称:BiTreeDepth(T)

初始条件:二叉树T已存在

操作结果:返回T的深度

实现思想:采用递归的思想,每进行一次递归,就将答案+1,比较左右两边子树的深度,返回深度更大的子树。

\subsubsection{查找结点}

函数名称:LocateNode(T,e)

初始条件:二叉树T已存在,e是T中某个结点关键字

操作结果:返回该结点指针

实现思想:采用递归地思想,依次比较结点的关键字和e的值,相同则返回该结点指针,不相同则继续向下寻找,直到到了空结点。如果左边答案指针leftResult不为空则返回leftResult,否则返回rightResult。

\subsubsection{查找父结点}

函数名称:LocateParent(T,e,k)

初始条件:二叉树T已存在,e是T中某个结点关键字

操作结果:返回父结点指针

实现思想:首先用LocateNode函数找到结点e,然后采用递归的思想依次从头结点向下寻找,直到Node结点的孩子是e结点,则返回Node结点并一步步退出递归,如图\ref{2-6}。若到空结点则返回NULL。

\begin{figure}[htb]
	\begin{center}
		\includegraphics[scale=0.5]{2-6.png}
		\caption{退出递归操作}
            \label{2-6}
	\end{center}
\end{figure}

\subsubsection{获得兄弟结点}

函数名称:LocateBrother(T,e)

初始条件:二叉树T已存在,e是T中某个结点关键字

操作结果:返回值为兄弟结点的指针

实现思想:有了前两个函数的支撑,该函数就迎刃而解。依次找到结点e以及父结点Parent,返回不等于e的另一个孩子结点指针。

\subsubsection{求结点深度}

函数名称:NodeDepth(T,e)

初始条件:二叉树T已存在,e是T中某个结点关键字

操作结果:返回Node结点的深度

实现思想:用LocateNode结点找到e,然后循环调用LocateParent,并记录次数,直到找到头结点为止。

\subsubsection{判断是否重复}

函数名称:UnExistNode(T,value)

初始条件:二叉树已存在

操作结果:若value在T中存在,则返回ERROR,否则返回OK

实验思想:这里用到了ans的乘法,如图\ref{2-4},只要出现依次关键字重复的情况,ans就=0,导致赋值失败。

\begin{figure}[htb]
	\begin{center}
		\includegraphics[scale=0.5]{2-4.png}
		\caption{ans的赋值操作}
            \label{2-4}
	\end{center}
\end{figure}

\subsubsection{结点赋值}

函数名称:Assign(\&T,e,value)

初始条件:二叉树T已存在,e是某个结点关键字,value的关键字与T不重复

操作结果:用value代替e的结点数据

实现思想:在最外层递归下调用函数LocateNode找到结点e,然后调用函数UnExistNode判断value关键字与T是否重复,依次退出递归,在最外层递归函数中完成赋值操作,如图\ref{2-5}。

\begin{figure}[htb]
	\begin{center}
		\includegraphics[scale=0.7]{2-5.png}
		\caption{最外层函数赋值}
            \label{2-5}
	\end{center}
\end{figure}

\subsubsection{插入结点}

函数名称:InsertNode(\&T,e,LR,c)

初始条件:二叉树T已存在,e是T中某个结点关键字,c中的关键字与T不重复

操作结果:LR为0或1,插入c为e的左子树或右子树,e原有的左子树或右子树变为c的右子树

实验思想:LR=-1即在头结点插入新结点,比较简单。若LR=0或1,首先找到e结点,然后根据LR的值进行插入,如图\ref{2-7}。

\begin{figure}[htb]
	\begin{center}
		\includegraphics[scale=0.7]{2-7.png}
		\caption{LR讨论}
            \label{2-7}
	\end{center}
\end{figure}

\subsubsection{删除结点}

函数名称:DeleteNode(\&T,e,k)

初始条件:二叉树T已存在,e是T中某个结点关键字

操作结果:e是叶子结点,直接删除;只有左子树或右子树,用其左子树或右子树代替e的位置;既有左子树又有右子树,将左子树代替e,右子树插入到左子树的最右结点的右子树

实现思想:当e是头结点时,可以看作递归自己,然后和其他结点的情况合并到下面进行操作,如图\ref{2-8}。

\begin{figure}[htb]
	\begin{center}
		\includegraphics[scale=0.6]{2-8.png}
		\caption{e为头结点递归自己}
            \label{2-8}
	\end{center}
\end{figure}

当e没有孩子时,直接释放e空间并返回2;

若e左右孩子都存在时,先将指向e的指针T指向e最右结点,然后返回3;

若e只有右孩子,则返回4;

若e只有左孩子,次情况可等同于最有结点,所以返回3。

退出最内层递归后,T指向的是e的父结点。

对于返回2的情况,我们只需将指向e的指针指向空;

对于返回3的情况,需要将T的左子树变成e的左子树,再释放e的空间;

对于返回4的情况,需要将T的左子树变成e的右子树,再释放e的空间;

对于返回OK的情况,说明已经完成了释放空间的操作,只需要一步步退出递归,返回值为OK。

图\ref{2-9}只是描述了e是T的左孩子的情况,e是右孩子的情况同上。

\begin{figure}[htb]
	\begin{center}
		\includegraphics[scale=0.5]{2-9.png}
            \includegraphics[scale=0.6]{2-10.png}
		\caption{对不同情况需要进行的操作}
            \label{2-9}
	\end{center}
\end{figure}

\subsubsection{前序遍历}

函数名称:PreOrderTraverse(T)

初始条件:二叉树T已存在

操作结果:先序遍历二叉树T

实现思想:先调用访问函数visit,再依次对左、右孩子进行递归调用。

\subsubsection{中序遍历}

函数名称:InOrderTraverse(T)

初始条件:二叉树T已存在

操作结果:中序遍历二叉树T

实现思想:先对左孩子进行递归调用,然后调用访问函数visit、最后右孩子进行递归调用。

\subsubsection{后序遍历}

函数名称:PostOrderTraverse(T)

初始条件:二叉树T已存在

操作结果:后序遍历二叉树T

实现思想:先依次对左、右孩子进行递归调用,再调用访问函数visit。

\subsubsection{层序遍历}

函数名称:LevelOrderTraverse(T)

初始条件:二叉树T已存在

操作结果:层序遍历二叉树T

实现思想:需要定义一个先进先出的队列,依次从前到后访问结点,每访问一个结点就将左右孩子加入到队列中,然后根结点出队列。

\subsection{系统测试}

二叉树的演示系统如图\ref{2-12},二叉树的创建如下,在这个系统测试中我们默认使用如图\ref{2-2}的二叉树。

\begin{figure}[htb]
	\begin{center}
		\includegraphics[scale=0.7]{2-12.png}
		\caption{演示系统}
            \label{2-12}
	\end{center}
\end{figure}

\begin{figure}[htb]
	\begin{center}
            \includegraphics[scale=0.7]{2-1.png}
		\includegraphics[scale=0.7]{2-2.png}
		\caption{按definition创建的二叉树}
            \label{2-2}
	\end{center}
\end{figure}

用功能1创建该二叉树,创建成功。

用功能16查找5的父结点,得到3;查找1的父结点则失败。

用功能8获取5的兄弟结点,得到4。

用功能9在3的左孩子插入新的结点6,f,然后用功能14层序遍历,得到1,2,3,6,5,4。

用功能10删除结点3,然后用功能13后序遍历,得到2,5,4,6,1。

用功能15翻转二叉树,然后用功能11先序遍历,得到1,6,4,5,2。

用功能18获得2和6的最近祖先,得到1。

用功能19判断是否为二叉排序树,得到不是排序树。

用功能5求二叉树深度,得到4。

用功能2销毁二叉树,销毁成功。

用功能12中序遍历,什么也没得到。

用功能3清除二叉树,清除失败。

最后输出0,退出演示系统。

\begin{figure}[htb]
	\begin{center}
            \includegraphics[scale=0.6]{2-11.png}
            \includegraphics[scale=1.2]{2-13.png}
            \includegraphics[scale=1.1]{2-14.png}
            \includegraphics[scale=1]{2-15.png}
		\caption{系统测试}
            \label{2-2}
	\end{center}
\end{figure}

\begin{figure}[htb]
	\begin{center}
            \includegraphics[scale=1.1]{2-16.png}
            \includegraphics[scale=1.1]{2-17.png}
            \includegraphics[scale=1.1]{2-18.png}
            \includegraphics[scale=1.1]{2-19.png}
            \includegraphics[scale=1.2]{2-20.png}
            \includegraphics[scale=1.2]{2-21.png}
            \includegraphics[scale=1.2]{2-22.png}
            \includegraphics[scale=1.2]{2-23.png}
            \includegraphics[scale=1.1]{2-24.png}
            \includegraphics[scale=1.1]{2-25.png}
            \includegraphics[scale=1.1]{2-26.png}
		\caption{系统测试}
            \label{2-2}
	\end{center}
\end{figure}

\subsection{实验小结}

在二叉树实验中,我学到了很多关于二叉树数据结构和相关算法的知识,也提升了自己的编程能力和问题解决能力。以下是我对二叉树实验的小结:

1.理解了二叉树的基本概念和性质:通过实验,我深入理解了二叉树的定义、特点以及常见的遍历方式(前序、中序、后序遍历),掌握了二叉树的基本操作。

2.实现了二叉树各种遍历算法:在实验过程中,我编写了二叉树的前序、中序、后序遍历的递归和非递归算法,加深了对这些遍历算法的理解和应用。

3.解决了二叉树相关问题:在实验过程中,我遇到了一些二叉树相关的问题,如查找二叉树的最大深度、判断二叉树是否对称等,通过实践和调试,我提升了自己的问题解决能力。

4.加深了对递归思想的理解:二叉树的遍历算法和一些问题解决方法都涉及到递归思想,通过实验,我进一步理解了递归的原理和应用场景,提高了递归编程的能力。

5.编程技能得到提升:通过实验的编程实践,我熟练掌握了二叉树的实现和操作方法,提升了自己的数据结构和算法编程能力。

总的来说,通过二叉树实验,我不仅学到了关于二叉树的知识和算法,还提升了编程能力和解决问题的能力,这对我今后在计算机科学领域的学习和发展都有很大帮助。

\section*{课程的收获和建议}

收获:

1.理解了不同数据结构的特点和应用场景,如数组、链表、栈、队列、树、图等,能更好地选择合适的数据结构来解决问题。

2.学会了数据结构的基本操作和算法,如插入、删除、查找、排序、遍历等,提高了编程的效率和质量。

3.掌握了常用的数据结构和算法的实现方式,加深了对计算机程序设计的理解和实践能力。

4.增强了问题分析和解决能力,能够更快速地解决实际的编程问题和挑战。

建议:

1.增加实际应用案例:在教学过程中,可以增加一些实际的应用案例,让学生更直观地理解数据结构在实际项目中的应用和重要性。

2.强调算法分析和复杂度:在教学中,可以更加强调算法分析和时间复杂度、空间复杂度的重要性,让学生深刻认识到选择合适的数据结构和算法对程序性能的影响。

3.提供更多编程练习和实践机会:通过更多的编程作业和项目,让学生能够真正动手实践并应用数据结构,加深对知识的理解和掌握。

4.引入新的数据结构和算法:随着科技的发展,新的数据结构和算法不断涌现,可以引入一些新颖的数据结构和算法,激发学生的兴趣,同时也让他们了解最新的技术动态。

5.注重技术实践和项目实践:可以结合实际项目需求,组织一些数据结构的实践项目,让学生能够将所学知识应用到实际项目中,提高他们的实际能力和竞争力。

\newpage

\section{参考文献}

[1] 严蔚敏等.数据结构(C语言版).清华大学出版社

[2] Larry Nyhoff. ADTs, Data Structures, and Problem Solving with C++.  Second Edition,Calvin College,2005

[3] 殷立峰. Qt C++跨平台图形界面程序设计基础. 清华大学出版社,2014:192~197

[4] 严蔚敏等.数据结构题集(C语言版).清华大学出版社

\newpage

\nocite{*} %% 作用是不对文献进行引用,但可以生成文献列表

%\bibliographystyle{HustGraduPaper}
%\bibliography{HustGraduPaper}

\section*{附录A:基于顺序存储结构线性表实现的源程序}

\begin{lstlisting}
#include <stdio.h>
#include <malloc.h>
#include <stdlib.h>

#define TRUE 1
#define FALSE 0
#define OK 1
#define ERROR 0
#define INFEASIBLE -1
#define OVERFLOW -2

typedef int status;
typedef int ElemType; //数据元素类型定义

#define LIST_INIT_SIZE 100
#define LISTINCREMENT  10

typedef struct { //顺序表(顺序结构)的定义
	ElemType *elem;
	int length;
	int listsize;
} SqList;

status InitList(SqList &L) {
	if (L.elem != NULL) {
		return INFEASIBLE;
	}
	L.elem = (ElemType*)malloc(LIST_INIT_SIZE*sizeof(ElemType));
	L.length = 0;
	L.listsize = LIST_INIT_SIZE;
	return OK;
}

status DestroyList(SqList &L)
// 如果线性表L存在,销毁线性表L,释放数据元素的空间,
返回OK,否则返回INFEASIBLE。
{
	if (L.elem == NULL) {
		return INFEASIBLE;
	}
	L.elem = NULL;
	free(L.elem);
	return OK;
}

status ClearList(SqList &L)
// 如果线性表L存在,删除线性表L中的所有元素,
返回OK,否则返回INFEASIBLE。
{
	if (L.elem == NULL) {
		return INFEASIBLE;
	}
	for (int i = 0; i < L.length; i++) {
		L.elem[i] = 0;
	}
	L.length = 0;
	return OK;
}

status ListEmpty(SqList L)
// 如果线性表L存在,判断线性表L是否为空,空就返回TRUE,
否则返回FALSE;如果线性表L不存在,返回INFEASIBLE。
{
	if (L.elem == NULL)
		return INFEASIBLE;
	if (L.length == 0)
		return TRUE;
	else
		return FALSE;
}

status ListLength(SqList L)
// 如果线性表L存在,返回线性表L的长度,否则返回INFEASIBLE。
{
	if (L.elem == NULL)
		return INFEASIBLE;
	return L.length;
}

status GetElem(SqList L, int i, ElemType &e)
// 如果线性表L存在,获取线性表L的第i个元素,保存在e中,
返回OK;如果i不合法,返回ERROR;
如果线性表L不存在,返回INFEASIBLE。
{
	if (L.elem == NULL) {
		return INFEASIBLE;
	}
	if (i > L.length || i < 1) {
		return ERROR;
	}
	e = L.elem[i - 1];
	return OK;
}

int LocateElem(SqList L, ElemType e)
// 如果线性表L存在,查找元素e在线性表L中的位置序号并返回该序号;
如果e不存在,返回0;当线性表L不存在时,返回INFEASIBLE(即-1)。
{
	if (L.elem == NULL) {
		return INFEASIBLE;
	}
	for (int i = 1; i <= L.length; i++) {
		if (e == L.elem[i - 1]) {
			return i;
		}
	}
	return 0;
}

status PriorElem(SqList L, ElemType e, ElemType &pre)
// 如果线性表L存在,获取线性表L中元素e的前驱,保存在pre中,
返回OK;如果没有前驱,返回ERROR;如果线性表L不存在,返回INFEASIBLE。
{
	if (L.elem == NULL) {
		return INFEASIBLE;
	}
	for (int i = 1; i <= L.length; i++) {
		if (e == L.elem[i - 1]) {
			if (i == 1)
				return ERROR;
			else {
				pre = L.elem[i - 2];
				return OK;
			}
		}
	}
	return ERROR;
}

status NextElem(SqList L, ElemType e, ElemType &next)
// 如果线性表L存在,获取线性表L元素e的后继,保存在next中,返回OK;
如果没有后继,返回ERROR;如果线性表L不存在,返回INFEASIBLE。
{
	if (L.elem == NULL) {
		return INFEASIBLE;
	}
	for (int i = 1; i <= L.length; i++) {
		if (e == L.elem[i - 1]) {
			if (i == L.length)
				return ERROR;
			else {
				next = L.elem[i];
				return OK;
			}
		}
	}
	return ERROR;
}

status ListInsert(SqList &L, int i, ElemType e)
// 如果线性表L存在,将元素e插入到线性表L的第i个元素之前,返回OK;
当插入位置不正确时,返回ERROR;如果线性表L不存在,返回INFEASIBLE。
{
	if (L.elem == NULL) {
		return INFEASIBLE;
	}
	if (i < 1 || i > L.length + 1) {
		return ERROR;
	}
	(int *)realloc(L.elem, sizeof(int *) * 11);
	int x;

	for (int x = L.length; x >= i; x--) {
		L.elem[x] = L.elem[x - 1];
	}
	L.elem[i - 1] = e;
	L.length++;
	return OK;
}

status ListDelete(SqList &L, int i, ElemType &e)
// 如果线性表L存在,删除线性表L的第i个元素,并保存在e中,返回OK;
当删除位置不正确时,返回ERROR;如果线性表L不存在,返回INFEASIBLE。
{
	if (L.elem == NULL) {
		return INFEASIBLE;
	}
	if (i < 1 || i > L.length) {
		return ERROR;
	}
	e = L.elem[i - 1];
	for (int x = i - 1; x < L.length - 1; x++) {
		L.elem[x] = L.elem[x + 1];
	}
	L.length--;
	return OK;
}

status ListTraverse(SqList L)
// 如果线性表L存在,依次显示线性表中的元素,每个元素间空一格,返回OK;
如果线性表L不存在,返回INFEASIBLE。
{
	if (L.elem == NULL) {
		return INFEASIBLE;
	}
	printf("遍历线性表:");
	for (int i = 0; i < L.length; i++) {
		printf("%d", L.elem[i]);
		if (i < L.length - 1)
			printf(" ");
	}
	return OK;
}

status MaxSubArray(SqList L) {
	if (L.elem == NULL) {
		return INFEASIBLE;
	}
	int sum = 0, result = 0;
	for (int i = 0; i < L.length; i++) {
		if (L.elem[i] >= 0) {
			sum += L.elem[i];
		} else {
			if (sum > result) {
				result = sum;
			}
			sum = 0;
		}
	}
	if (sum > result) {
		result = sum;
	}
	return result;
}

status SubArrayNum(SqList L, int k) {
	if (L.elem == NULL) {
		return INFEASIBLE;
	}
	int count = 0;
	for (int i = 0; i < L.length; i++) {
		int sum = 0;
		for (int j = i; j < L.length; j++) {
			sum += L.elem[j];
			if (sum == k)
				count++;
		}
	}
	return count;
}

status sortList(SqList &L) {
	if (L.elem == NULL) {
		return INFEASIBLE;
	}
	for (int i = 0; i < L.length - 1; i++) {
		for (int j = i + 1; j < L.length; j++) {
			if (L.elem[i] > L.elem[j]) {
				int temp = L.elem[i];
				L.elem[i] = L.elem[j];
				L.elem[j] = temp;
			}
		}
	}
	return OK;
}

status  SaveList(SqList L, char FileName[])
// 如果线性表L存在,将线性表L的的元素写到FileName文件中,
返回OK,否则返回INFEASIBLE。
{
	if (L.elem == NULL)
		return INFEASIBLE;
	FILE *fp = fopen(FileName, "w");
	if (fp == NULL) {
		printf("Error opening file\n");
		return 1;
	}

	for (int i = 0; i < L.length; i++) {
		fprintf(fp, "%d", L.elem[i]);
		fprintf(fp, " ");
	}
	fclose(fp);
	return OK;
}

status  LoadList(SqList &L, char FileName[])
// 如果线性表L不存在,将FileName文件中的数据读入到线性表L中,
返回OK,否则返回INFEASIBLE。
{
	if (L.elem != NULL) {
		return INFEASIBLE;
	}
	L.elem = (ElemType *)malloc(sizeof(ElemType) * L.listsize);
	FILE *fp = fopen(FileName, "r");
	if (fp == NULL) {
		printf("Error opening file\n");
		return 1;
	}

	for (int i = 0; i < L.length; i++) {
		fscanf(fp, "%d", &L.elem[i]);
	}
	fclose(fp);
	return OK;
}

int main( ) {
	SqList L;
	L.elem = NULL;
	int op = 1;
	int j;
	ElemType i, e, pre, next;
	while (op) {
	system("cls");
	printf("\n");
	printf("    Menu for Linear Table On Sequence Structure \n");
	printf("----------------------------------------------\n");
	printf("        1. InitList(线性表的创建)\n");
	printf("        2. DestroyList(销毁线性表)\n");
	printf("    	3. ClearList(清空线性表)\n");
	printf("        4. ListEmpty(线性表判空)\n");
	printf("    	5. ListLength(求线性表长度)\n");
	printf("        6. GetElem(获取元素)\n");
	printf("    	7. LocateElem(查找元素)\n");
	printf("        8. PriorElem(获取前驱元素)\n");
	printf("    	9. NextElem(获取后继元素)\n");
	printf("        10.ListInsert(插入元素)\n");
	printf("    	11.ListDelete(删除元素)\n");
	printf("        12.ListTrabverse(遍历线性表)\n");
	printf("        13.MaxSubArray(获取最大和)\n");
	printf("        14.SubArrayNum(求和个数)\n");
	printf("        15.sortList(线性表排序)\n");
	printf("        0. Exit(退出菜单)\n");
	printf("----------------------------------------------\n");
	printf("    请选择你的操作[0~12]:");
	scanf("%d", &op);		
        switch (op) {
	case 1:
		j = InitList(L);
		if ( j == OK)
			printf("线性表创建成功!\n");
		else
			printf("线性表创建失败!\n");
		getchar();
		getchar();
		break;
	case 2:
		j = DestroyList(L);
		if ( j == OK)
			printf("线性表销毁成功!\n");
		else
			printf("线性表销毁失败!\n");
		getchar();
		getchar();
		break;
	case 3:
		j = ClearList(L);
		if (j == OK)
			printf("线性表清空成功!\n");
		else
			printf("线性表清空失败!\n");
		getchar();
		getchar();
		break;
	case 4:
		j = ListEmpty(L);
		if (j == TRUE)
			printf("线性表为空!\n");
		else if (j == FALSE)
			printf("线性表不为空!\n");
		else
			printf("线性表不存在!\n");
		getchar();
		getchar();
		break;
	case 5:
		j = ListLength(L);
		if ( j == INFEASIBLE)
			printf("线性表不存在!\n");
		else
			printf("线性表长度为%d\n", ListLength(L));
		getchar();
		getchar();
		break;
		case 6:
		printf("输入要获取的元素的位置\n");
		scanf("%d", &i);
		j = GetElem(L, i, e);
		if ( j == INFEASIBLE)
			printf("线性表不存在!\n");
		else if (j == ERROR)
			printf("获取元素失败!\n");
		else
			printf("链表中第%d个元素是%d\n", i, e);
		getchar();
		getchar();
		break;
	case 7:
		printf("请输入要查找的元素\n");
		scanf("%d", &e);
		j = LocateElem(L, e);
		if (j == 0)
			printf("该元素不存在!\n");
		else
			printf("该元素的位置序号是%d\n", j);
		getchar();
		getchar();
		break;
	case 8:
		printf("请输入要查找的元素\n");
		scanf("%d", &e);
		j = PriorElem(L, e, pre);
		if (j == ERROR)
			printf("查找失败,%d为第一个元素或%d不存在!\n",
                e, e);
		else
			printf("%d的前驱元素是%d\n", e, pre);
		getchar();
		getchar();
		break;
	case 9:
		printf("请输入要查找的元素\n");
		scanf("%d", &e);
		j = NextElem(L, e, next);
		if (j == ERROR)
			printf("查找失败,%d为最后一个
                    元素或%d不存在!\n", e, e);
		else
			printf("%d的后继元素是%d\n", e, next);
		getchar();
		getchar();
		break;
	case 10:
		printf("输入要插入的元素的位置以及元素值\n");
		scanf("%d %d", &i, &e);
		j = ListInsert(L, i, e);
		if (j == OK)
			printf("插入成功!\n");
		else
			printf("插入失败!\n");
		getchar();
		getchar();
		break;
	case 11:
		printf("请输入要删除元素的位置\n");
		scanf("%d", &i);
		j = ListDelete(L, i, e);
		if (j == ERROR)
			printf("删除失败!\n");
		else
			printf("删除成功!\n");
		getchar();
		getchar();
		break;
	case 12:
		if (ListTraverse(L) == INFEASIBLE)
			printf("线性表不存在!\n");
		getchar();
		getchar();
		break;
	case 13:
		j = MaxSubArray(L);
		printf("最大和为%d\n", j);
		getchar();
		getchar();
		break;
	case 14:
		printf("请输入连续子数组的和\n");
		scanf("%d", &i);
		j = SubArrayNum(L, i);
		printf("该数组中和为%d的连续子数组的个数为%d\n", i, j);
		getchar();
		getchar();
		break;
	case 15:
		sortList(L);
		getchar();
		getchar();
		break;
	case 0:
		break;
	}//end of switch
	}//end of while
	printf("欢迎下次再使用本系统!\n");
	return 0;
}//end of main()
\end{lstlisting}

\section*{附录B:基于链式存储结构线性表实现的源程序}

\begin{lstlisting}
#include <stdio.h>
#include <malloc.h>
#include <stdlib.h>
#define OK 1
#define ERROR 0

typedef int status;
typedef int ElemType; //数据元素类型定义

typedef struct LNode { //单链表(链式结构)结点的定义
	ElemType data;
	struct LNode *next;
} LNode, *LinkList;

status InitList(LinkList &L)
//初始条件:线性表不存在
//操作结果:构造一个空的线性表L
{
	if (L != NULL) { //线性表不为空则返回ERROR
		return ERROR;
	}
	L = (LinkList)malloc(sizeof(LNode)); //动态分配空间
	L->data = 0; //初始值设为0;
	L->next = NULL; //空表的首结点指向空
	return OK;
}

status DestroyList(LinkList &L)
//初始条件:线性表L已存在
//操作结果:删除线性表
{
	if (L == NULL) { //线性表不存在
		return ERROR;
	}
	LinkList ne, q = L->next; //前后结点
	while (q) { //前结点不为空
		ne = q->next; //后结点赋值
		free(q); //删除前结点
		q = ne;
	}
	free(L); //将线性表释放空间
	L = NULL; //将线性表释放空间
	return OK;
}

status ClearList(LinkList &L)
//初始条件:线性表L已存在
//操作结果:将L重置为空表
{
	if (L == NULL) { //线性表不存在
		return ERROR;
	}
	LinkList p = L->next, ne; //前后结点
	while (p) { //前结点不为空
		ne = p->next; //后结点赋值
		free(p); //删除前结点
		p = ne;
	}
	L->next = NULL; //将线性表首结点指向空
	return OK;
}

status ListEmpty(LinkList L)
//初始条件:线性表L已存在
//操作结果:若L为空表,返回OK,否则返回ERROR
{
	if (L == NULL) { //线性表不存在
		return ERROR;
	}
	if (L->next == NULL) { //首结点为空
		return OK;
	}
	return ERROR;
}

int ListLength(LinkList L)
//初始条件:线性表L已存在
//操作结果:返回L中数据元素的个数
{
	if (L == NULL) { //线性表不存在
		return -1;
	}
	int len = 0; //初始化长度
	LinkList p = L->next;
	while (p) { //未到尾指针则循环
		len++;
		p = p->next;
	}
	return len;
}

status GetElem(LinkList L, int i, ElemType &e)
//初始条件:线性表L已存在,1<=i<=ListLength(L)
//操作结果:用e返回L中的第i个元素的值
{
	if (L == NULL) {
		return ERROR;
	}
	if (i < 1 || i > ListLength(L)) { //i的值不正确,返回ERROR
		return ERROR;
	}
	LinkList p = L;

	while (i > 0) { //找到第i个元素
		i--;
		p = p->next;
	}
	e = p->data; //将第i个元素赋值给e
	return OK;
}

status LocateElem(LinkList L, ElemType e)
//初始条件:线性表L已存在
//操作结果:返回e的结点位置
{
	if (L == NULL) { //线性表不存在
		return ERROR;
	}
	int i = 1;
	LinkList p = L->next;
	while (p) { //未到尾指针
		if (p->data == e) //找到了待找元素则返回结点位置
			return i;
		i++;
		p = p->next;
	}
	return ERROR; //未找到结点位置返回ERROR
}

status PriorElem(LinkList L, ElemType e, ElemType &pre)
//初始条件:线性表L已存在
//操作结果:获取线性表L中元素e的前驱,保存在pre中,返回OK;
如果没有前驱,返回ERROR
{
	if (L == NULL) { //线性表不存在
		return ERROR;
	}
	LinkList p = L, ne = L->next; //前后结点
	while (ne) {
		if (e == ne->data) { //找到了待找元素
			if (p == L) //e是第一个元素,没有前驱
				return ERROR;
			pre = p->data; //将前驱信息赋值给pre
			return OK;
		}
		p = ne;
		ne = p->next;
	}
	return ERROR; //未找到元素e
}

status NextElem(LinkList L, ElemType e, ElemType &next)
//初始条件:线性表L已存在
//操作结果:获取线性表L中元素e的后继,保存在next中,返回OK;
如果没有后继,返回ERROR
{
	if (L == NULL) { //线性表不存在
		return ERROR;
	}
	LinkList p = L->next, ne;
	if (p == NULL) //线性表是空表
		return ERROR;
	ne = p->next;
	while (ne) { //后元素存在
		if (e == p->data) { //找到了待找元素
			next = ne->data;
			return OK;
		}
		p = ne;
		ne = p->next;
	}
	return ERROR; //未找到元素e
}

status ListInsert(LinkList &L, int i, ElemType e)
//初始条件:线性表L已存在,1<=i<=ListLength(L)+1
//操作结果:在L中第i个位置之前插入新的元素数据e,L的长度+1
{
	if (L == NULL) { //线性表不存在
		return ERROR;
	}
	if (i < 1 || i > ListLength(L) + 1) { //i的值不正确,返回ERROR
		return ERROR;
	}
	LinkList p = L, t = (LinkList)malloc(sizeof(LNode)); 
        //动态分配一个新空间
	t->data = e;
	while (i-- > 1) { //寻找位置i
		p = p->next;
	}
	t->next = p->next; //修改指针位置
	p->next = t;
	return OK;
}

status ListDelete(LinkList &L, int i, ElemType &e)
//初始条件:线性表L已存在且非空,1<=i<=ListLength(L)
//操作结果:删除L中第i个数据元素,用e返回其值,L的长度-1
{

	if (L == NULL) { //线性表不存在
		return ERROR;
	}
	if (i < 1 || i > ListLength(L)) { //i的值不正确,返回ERROR
		return ERROR;
	}
	LinkList p = L;
	while (i-- > 1) { //寻找位置i的前驱结点
		p = p->next;
	}
	LinkList t = p->next; //t为位置i的结点指针
	if (t == NULL) {
		return ERROR;
	}
	e = t->data;
	p->next = t->next;
	free(t);
	return OK;
}

status ListTraverse(LinkList L)
//初始条件:线性表L已存在
//操作结果:依次访问L中的每个元素
{
	if (L == NULL) { //线性表不存在
		return ERROR;
	}
	LinkList p = L->next;
	while (p != NULL) {
		printf("%d ", p->data);
		p = p->next;
	}
	return OK;
}

status reverseList(LinkList &L) {
//初始条件:线性表L已存在
//操作结果:将L翻转
	if (L == NULL) {  //线性表不存在
		return ERROR;
	}
	LinkList p1 = L->next, p2 = p1->next, p3 = p1;
	p1->next = NULL; //首先将第一个元素置于末尾
	while (p2 != NULL) {
		p1 = p2;
		p2 = p2->next;
		p1->next = p3;
		p3 = p1;
	}
	L->next = p1; //将最后一个元素置于开头
	return OK;
}

status RemoveNthFromEnd(LinkList L, int n) {
//初始条件:线性表L已存在且非空
//操作结果:该链表中倒数第n个结点
	if (L == NULL) { //线性表不存在
		return ERROR;
	}
	int len = ListLength(L);
	if (n > len || n < 1) { //n的值不正确
		return ERROR;
	}
	LinkList p = L->next;
	for (int i = 0; i < len - n; i++) { //找到倒数第n个结点
		p = p->next;
	}
	return p->data;
}

status sortList(LinkList &L) {
//初始条件:线性表L已存在
//操作结果:将L由小到大排序
	if (L == NULL) { //线性表不存在
		return ERROR;
	}
	int flag = 0; //快速退出功能
	LinkList p1 = L, p2 = p1->next;
	while (1) { //采用冒泡排序的方法
		flag = 0; //还原flag的值

		while (p2->next != NULL) {
			if (p2->data > p2->next->data) {
				flag = 1; //表示有元素的位置要改变
				p1->next = p2->next; //实现两指针翻转
				p2->next = p1->next->next;
				p1->next->next = p2;
				p2 = p1->next;
			}
			p1 = p2;
			p2 = p2->next;
		}
		if (flag == 0) //已经完成排序操作,直接返回OK,节省时间
			break;
		p1 = L; //一趟排序结束后重新回到头结点进行下一趟排序
		p2 = p1->next;
	}
	return OK;
}

int main() {
	LinkList L = NULL;
	int op = 1;
	int j;
	ElemType i, e, pre, next;
while (op) {
	system("cls");
	printf("\n");
	printf("   Menu for Linear Table On Sequence Structure \n");
	printf("-----------------------------------------------\n");
	printf("       1. InitList(线性表的创建)\n");
	printf("       2. DestroyList(销毁线性表)\n");
	printf("       3. ClearList(清空线性表)\n");
	printf("       4. ListEmpty(线性表判空)\n");
	printf("       5. ListLength(求线性表长度)\n");
	printf("       6. GetElem(获取元素)\n");
	printf("       7. LocateElem(查找元素)\n");
	printf("       8. PriorElem(获取前驱元素)\n");
	printf("       9. NextElem(获取后继元素)\n");
	printf("       10.ListInsert(插入元素)\n");
	printf("       11.ListDelete(删除元素)\n");
	printf("       12.ListTrabverse(遍历线性表)\n");
	printf("       13.reverseList(链表翻转)\n");
	printf("       14.RemoveNthFromEnd(倒数第n个结点)\n");
	printf("       15.sortList(链表排序)\n");
	printf("       0. Exit(退出菜单)\n");
	printf("----------------------------------------------\n");
	printf("   请选择你的操作[0~15]:");
	scanf("%d", &op);
	switch (op) {
	case 1:
		j = InitList(L);
		if (j == OK)
			printf("线性表创建成功!\n");
		else
			printf("线性表创建失败!\n");
		getchar();
		getchar();
		break;
	case 2:
		j = DestroyList(L);
		if (j == OK)
			printf("线性表销毁成功!\n");
		else
			printf("线性表销毁失败!\n");
		getchar();
		getchar();
		break;
	case 3:
		j = ClearList(L);
		if (j == OK)
			printf("线性表清空成功!\n");
		else
			printf("线性表清空失败!\n");
		getchar();
		getchar();
		break;
	case 4:
		j = ListEmpty(L);
		if ( j == OK)
			printf("线性表为空!\n");
		else if (j == ERROR)
			printf("线性表不为空!\n");
		else
			printf("线性表不存在!\n");
		getchar();
		getchar();
		break;
	case 5:
		j = ListLength(L);
		if (j == -1)
		printf("线性表不存在!\n");
		else
	       	printf("线性表长度为%d\n", ListLength(L));
		getchar();
		getchar();
		break;
	case 6:
		printf("输入要获取的元素的位置\n");
		scanf("%d", &i);
		j = GetElem(L, i, e);
		if (j == ERROR)
		printf("获取元素失败!\n");
		else
		printf("链表中第%d个元素是%d\n", i, e);
		getchar();
		getchar();
		break;
	case 7:
		printf("请输入要查找的元素\n");
		scanf("%d", &e);
		j = LocateElem(L, e);
		if (j == ERROR)
			printf("该元素不存在!\n");
		else
			printf("该元素的位置序号是%d\n", j);
		getchar();
		getchar();
		break;
	case 8:
		printf("请输入要查找的元素\n");
		scanf("%d", &e);
		j = PriorElem(L, e, pre);
		if (j == ERROR)
			printf("查找失败\n");
		else
			printf("%d的前驱元素是%d\n", e, pre);
		getchar();
		getchar();
		break;
	case 9:
		printf("请输入要查找的元素\n");
		scanf("%d", &e);
		j = NextElem(L, e, next);
		if (j == ERROR)
			printf("查找失败\n");
		else
			printf("%d的后继元素是%d\n", e, next);
		getchar();
		getchar();
		break;
	case 10:
		printf("输入要插入的元素的位置以及元素值\n");
		scanf("%d %d", &i, &e);
		j = ListInsert(L, i, e);
		if (j == OK)
			printf("插入成功!\n");
		else
			printf("插入失败!\n");
		getchar();
		getchar();
		break;
	case 11:
		printf("请输入要删除元素的位置\n");
		scanf("%d", &i);
		j = ListDelete(L, i, e);
		if (j == ERROR)
			printf("删除失败!\n");
		else
			printf("删除成功!\n");
		getchar();
		getchar();
		break;
	case 12:
	      ListTraverse(L);
		getchar();
		getchar();
		break;
	case 13:
		reverseList(L);
		printf("翻转成功\n");
		getchar();
		getchar();
		break;
	case 14:
		printf("请输入想要获得的倒数第n个元素\n");
		scanf("%d", &i);
		j = RemoveNthFromEnd(L, i);
		if (j == ERROR)
			printf("n的值错误\n", i);
		else
			printf("倒数第%d个元素为%d\n", i, j);
		getchar();
		getchar();
		break;
	case 15:
		j = sortList(L);
		printf("排序成功\n");
		getchar();
		getchar();
		break;
	case 0:
		break;
	}//end of switch
}//end of while
printf("欢迎下次再使用本系统!\n");
return 0;
}//end of main()
\end{lstlisting}

\section*{附录C:基于顺序存储结构线性表实现的源程序}

\begin{lstlisting}
#include <stdio.h>
#include <stdlib.h>
#include <string.h>
#define OK 1
#define ERROR 0

typedef int status;
typedef int KeyType;

typedef struct {  //二叉树结点类型定义
	KeyType key;
	char others[20];
} TElemType;

typedef struct BiTNode { //二叉链表结点的定义
	TElemType data;
	struct BiTNode *lchild, *rchild;
} BiTNode, *BiTree;

void visit(BiTree T) { //二叉树访问
	printf("%d,%s ", T->data.key, T->data.others);
}

status CreateBiTree(BiTree &T, TElemType definition[], int begin) {
//初始条件:二叉树已存在
//操作结果:按definition构造二叉树T
	for (int i = 0; definition[i].key != -1; i++) {
            //检查是否有关键字相同的结点
		for (int j = i + 1; definition[j].key != -1; j++) {
			if (definition[i].key == definition[j].key &&
                    definition[i].key != 0) {
				return ERROR;
			}
		}
	}
	static int k = 0;//静态变量保持在递归中k的值可以传回上层函数
	if (definition[k].key == -1) {
		return OK;
	}
	T = (BiTree)malloc(sizeof(BiTNode));
	if (definition[k].key != 0) {//下一个结点不为空
		T->data.key = definition[k].key;
		strcpy(T->data.others, definition[k].others);
		k++;
		CreateBiTree(T->lchild, definition, begin + 1);
		CreateBiTree(T->rchild, definition, begin + 1);
	} else {           //下一结点为空
		k++;
		T = NULL;
	}
	if (begin == 1)
		k = 0;
	return OK;
}

status DestroyBiTree(BiTree &T) {
//初始条件:二叉树T已存在
//操作结果:销毁二叉树T
	if (T == NULL)
		return OK;
	DestroyBiTree(T->lchild);
	DestroyBiTree(T->rchild);
	free(T);
	T = NULL;
	return OK;
}

status ClearBiTree(BiTree &T) {
//初始条件:二叉树T已存在
//操作结果:将二叉树T清空为空树
	if (T == NULL)
		return OK;
	ClearBiTree(T->lchild);
	ClearBiTree(T->rchild);
	T->data.key = 0;
	strcpy(T->data.others, "\0");
	T->lchild = NULL;
	T->rchild = NULL;
	return OK;
}

status BiTreeEmpty(BiTree T) {
//初始条件:二叉树T已存在
//操作结果:若T为空二叉树,则返回OK,否则返回ERROR
	if (T == NULL)
		return OK;
	else
		return ERROR;
}

int BiTreeDepth(BiTree T) {
//初始条件:二叉树T已存在
//操作结果:返回T的深度
	if (T == NULL) {
		return 0;
	}
	int ldeep = 1 + BiTreeDepth(T->lchild);
	int rdeep = 1 + BiTreeDepth(T->rchild);
	if (ldeep > rdeep)
		return ldeep;
	else
		return rdeep;
	/********** End **********/
}

BiTNode *LocateNode(BiTree T, KeyType e) {
//初始条件:二叉树T已存在,e是T中某个结点关键字
//操作结果:返回该结点指针
	if (T == NULL) { //已经到了叶子结点,就返回空结点
		return NULL;
	}
	if (T->data.key == e) { //找到了,则返回该结点
		return T;
	} else { //没有找到,继续向下寻找
		BiTNode *leftResult = LocateNode(T->lchild, e);
		BiTNode *rightResult = LocateNode(T->rchild, e);
		if (leftResult != NULL) { //在左子树中说明找到了
			return leftResult; //返回刚刚找到的结点
		} else { 
			return rightResult;
		}
	}
}

BiTNode *LocateParent(BiTree T, KeyType e, int k) {
//初始条件:二叉树T已存在,e是T中某个结点关键字
//操作结果:返回父结点指针
	if (T == NULL)
		return NULL;
	if (T->data.key == e) {
		if (k == 1) //k=1就是头结点,k>1就不是头结点
			return NULL;
		else
			return T;
	}
	BiTree Node = LocateNode(T, e);
	if (Node == NULL) //e的值不正确
		return NULL;
	BiTree leftNode = LocateParent(T->lchild, e, k + 1);
	if (leftNode != NULL) { //e在T的左子树中
		if (T->lchild == Node) //T的左孩子就是e,返回T
			return T;
		else
			return leftNode;
	}
	BiTree rightNode = LocateParent(T->rchild, e, k + 1);
	if (rightNode != NULL) { //e在T的右子树中
		if (T->rchild == Node) //T的右孩子就是e,返回T
			return T;
		else
			return rightNode;
	}
	return NULL;
}

BiTNode *LocateBrother(BiTree T, KeyType e) {
//初始条件:二叉树T已存在,e是T中某个结点关键字
//操作结果:返回值为兄弟结点的指针
	BiTree ParentNode = LocateParent(T, e, 1);
	if (!ParentNode)//e的父结点不存在
		return NULL;
	BiTree Node = LocateNode(T, e); //e一定存在且不为头结点
	if (Node == ParentNode->lchild)
		return ParentNode->rchild;
	else
		return ParentNode->lchild;
}

int NodeDepth(BiTree T, KeyType e) {
//初始条件:二叉树T已存在,e是T中某个结点关键字
//操作结果:返回Node结点的深度
	if (T->data.key == e)
		return 1;
	BiTree ParentNode = LocateNode(T, e); //找到e的父结点
	if (ParentNode == NULL) //e是头结点
		return ERROR;
	int deep = 1;
	do {
		ParentNode = LocateParent(T, ParentNode->data.key, 1);
		deep++;
	} while (T != ParentNode); //循环求父结点,直到结点等于头结点
	return deep;
}

status UnExistNode(BiTree T, KeyType value) {
//初始条件:二叉树已存在
//操作结果:若value在T中存在,则返回ERROR,否则返回OK
	if (T == NULL) { //到了空结点
		return OK;
	}
	if (T->data.key == value) { //关键字存在
		return ERROR; //赋值失败,返回0
	}
	int ans = 1 * UnExistNode(T->lchild, value) * 
                UnExistNode(T->rchild, value);
	//采用乘法,只要关键字存在就失败
	if (ans == 0) //关键字存在
		return ERROR;
	else //关键字不存在
		return OK;
}


status Assign(BiTree &T, KeyType e, TElemType value, int k) {
//初始条件:二叉树T已存在,e是某个结点关键字,value的关键字与T不重复
//操作结果:用value代替e的结点数据
	BiTree Node = LocateNode(T, e); //找关键字为e的结点
	if (!Node) //e的值不正确,返回ERROR
		return ERROR;
	if (!UnExistNode(T,value.key)&&value.key!=Node->data.key){
                //关键字重复且不是e
		return ERROR; //赋值失败,返回0
	}
	Node->data = value;
	return OK;
}

status InsertNode(BiTree &T, KeyType e, int LR, TElemType c) {
//初始条件:二叉树T已存在,e是T中某个结点关键字,c中的关键字与T不重复
//操作结果:LR=0或1,插入c为e的左子树或右子树,e原有的左子树
或右子树变为c的右子树LR=-1插入到头结点,原二叉树变为c的右子树
	if (LR == -1) { //插入的结点是头结点
		BiTree temp = (BiTree)malloc(sizeof(BiTNode));
		temp->data = c;
		temp->lchild = NULL;
		temp->rchild = T; //原来树是新结点的右子树
		T = temp;
		return OK;
	}
	BiTree Node = LocateNode(T, e); //找到需要插入结点的父结点
	if (!Node) //e的值错误
		return 0;
	if (!UnExistNode(T, c.key) && c.key != Node->data.key) { 
                //关键字重复且不是e
		return ERROR; //赋值失败,返回0
	}
	BiTree temp = (BiTree)malloc(sizeof(BiTNode));
	temp->data = c;
	temp->lchild = NULL;
	if (LR == 0) { //左孩子
		temp->rchild = Node->lchild;
		Node->lchild = temp;
	} else if (LR == 1) { //右孩子
		temp->rchild = Node->rchild;
		Node->rchild = temp;
	}
	return OK;
}

status DeleteNode(BiTree &T, KeyType e, int k) {
//初始条件:二叉树T已存在,e是T中某个结点关键字
//操作结果:e是叶子结点,直接删除;只有左子树或右子树,用其左子树
或右子树代替e的位置;既有左子树又有右子树,将左子树代替e,右子树插
入到左子树的最右结点的右子树
	if (T == NULL)
		return 0;
	if (e == T->data.key) { //判断的结点就是需要删除的结点
		if (k == 1) { //k的作用是判断e是否为头结点
			int ans = DeleteNode(T, e, k + 1); 
                    //递归自己,假设头结点的孩子还是头结点
			if (ans == OK) { //头结点
				return OK;
			} else if (ans == 2) { //头结点就是孤立结点
				return OK;
			} else if (ans == 3) {
				BiTree temp = T;
				T = T->lchild;
				free(temp);
				temp = NULL;
				return OK;
			} else if (ans == 4) {
				BiTree temp = T;
				T = temp->rchild;
				free(temp);
				temp = NULL;
				return OK;
			}
		} else if (T->lchild == NULL && T->rchild == NULL) { 
                    //e是叶子结点
			free(T);
			T = NULL;
			return 2;
		} else if (T->lchild != NULL && T->rchild != NULL) { 
                    //e左右孩子都存在
			BiTree temp = T->lchild;
			while (temp->rchild) { 
                    //找到T左子树的最右侧,需要在后面接入T的右子树
				temp = temp->rchild;
			}
			temp->rchild = T->rchild;
			return 3;
		} else if (T->lchild == NULL) { //只有右孩子
			return 4;
		} else { //只有左孩子
			return 3;
		}
	}
	int lans = DeleteNode(T->lchild, e, k + 1); //左答案
	if (lans == OK) { //OK说明已经完成删除操作,只需一步步退出递归
		return OK;
	} else if (lans == 2) { //e是叶子结点,删除后T的左子树变为空
		T->lchild == NULL;
		return OK;
	} else if (lans == 3) { //需要将T的左子树变成e的左子树
		BiTree temp = T->lchild;
		T->lchild = temp->lchild;
		free(temp);
		temp = NULL;
		return OK;
	} else if (lans == 4) { //需要将T的左子树变成e的右子树
		BiTree temp = T->lchild;
		T->lchild = temp->rchild;
		free(temp);
		temp = NULL;
		return OK;
	} //在T的左子树中并未找到e
	int rans = DeleteNode(T->rchild, e, k + 1);
	if (rans == OK) {
		return OK;
	} else if (rans == 2) {
		T->rchild = NULL;
		return OK;
	} else if (rans == 3) {
		BiTree temp = T->rchild;
		T->rchild = temp->lchild;
		free(temp);
		temp = NULL;
		return OK;
	} else if (rans == 4) {
		BiTree temp = T->rchild;
		T->rchild = temp->rchild;
		free(temp);
		temp = NULL;
		return OK;
	}
	return 0;
}

status PreOrderTraverse(BiTree T) {
//初始条件:二叉树T已存在
//操作结果:先序遍历二叉树T
	if (T == NULL)
		return ERROR;
	visit(T);
	PreOrderTraverse(T->lchild);
	PreOrderTraverse(T->rchild);
	return OK;
}

status InOrderTraverse(BiTree T) {
//初始条件:二叉树T已存在
//操作结果:中序遍历二叉树T
	if (T == NULL)
		return ERROR;
	InOrderTraverse(T->lchild);
	visit(T);
	InOrderTraverse(T->rchild);
	return OK;
}

status PostOrderTraverse(BiTree T) {
//初始条件:二叉树T已存在
//操作结果:后序遍历二叉树T
	if (T == NULL)
		return ERROR;
	PostOrderTraverse(T->lchild);
	PostOrderTraverse(T->rchild);
	visit(T);
	return OK;
}

status LevelOrderTraverse(BiTree T) {
//初始条件:二叉树T已存在
//操作结果:层序遍历二叉树T
	if (T == NULL)
		return ERROR;
	BiTree Queue[20];
	int i = 0, k = 0;
	Queue[i++] = T;

	while (k < i) {
		BiTree current = Queue[k++];
		visit(current);
		if (current->lchild)
			Queue[i++] = current->lchild;
		if (current->rchild)
			Queue[i++] = current->rchild;
	}
	return OK;
}

status InvertTree(BiTree &T) {
//初始条件:二叉树T已存在
//操作结果:将二叉树T左右子树互换位置
	if (T == NULL)
		return OK;
	BiTree temp = T->lchild;
	T->lchild = T->rchild;
	T->rchild = temp;
	InvertTree(T->lchild);
	InvertTree(T->rchild);
	return OK;
}

BiTNode *LocateCommonAncestor(BiTree &T, KeyType e1, KeyType e2) {
//初始条件:二叉树T已存在
//操作结果:返回两个结点e1和e2的最近祖先
	BiTree Node1 = LocateNode(T, e1); //找到两个要求的结点
	BiTree Node2 = LocateNode(T, e2);
	if (Node1 == NULL || Node2 == NULL) //其中有空结点据返回NULL
		return NULL;
	int deep1 = NodeDepth(T, e1); //两个结点的深度
	int deep2 = NodeDepth(T, e2);

	while (Node1 != Node2) { //循环找到最近祖先则退出循环
		if (deep1 > deep2) {
			Node1 = LocateParent(T, Node1->data.key, 1);
			deep1--;
		} else if (deep1 <= deep2) {
			Node2 = LocateParent(T, Node2->data.key, 1);
			deep2--;
		}
	}
	return Node1;
}

static int k = 0;

status JudgeBiTree(BiTree T) {
//初始条件:二叉树T已存在
//操作结果:二叉树T是为二叉排序树则返回OK,否则返回ERROR
	//按照先序遍历整个树,k保证先遍历的结点小于后遍历的结点
	int ans = 1; //采用乘法,只要出现一次不符合二叉排序的
                结点就等于0,否则就等于1
	if (T == NULL) //找到空结点
		return OK;
	ans *= JudgeBiTree(T->lchild); //首先访问左子树中的结点

	if (T->data.key >= k) { //当前结点的数据大于之前访问的结点
		k = T->data.key;
	} else //不满足则直接返回0;
		return ERROR;
	ans *= JudgeBiTree(T->rchild);
	return ans; //成功返回1,失败返回0
}

int main( ) {
	BiTree T = NULL, getT = NULL;
	int op = 1, ans, i = 0, begin = 0, e, LR, e2;
	TElemType value;
	while (op) {
	system("cls");
	printf("\n");
	printf("    Menu for Linear Table On Sequence Structure \n");
	printf("----------------------------------------------\n");
	printf("    	1. CreateBiTree (创建二叉树)\n");
	printf("        2. DestroyBiTree (销毁二叉树)\n");
	printf("    	3. ClearBiTree (清空二叉树)\n");
	printf("        4. BiTreeEmpty (二叉树判空)\n");
	printf("    	5. BiTreeDepth (求二叉树深度)\n");
	printf("        6. LocateNode (查找结点)\n");
	printf("    	7. Assign (结点赋值)\n");
	printf("        8. LocateBrother (获取兄弟结点)\n");
	printf("    	9. InsertNode (插入结点)\n");
	printf("        10.DeleteNode (删除结点)\n");
	printf("    	11.PreOrderTraverse (前序遍历)\n");
	printf("        12.InOrderTraverse (中序遍历)\n");
	printf("        13.PostOrderTraverse (后序遍历)\n");
	printf("        14.LevelOrderTraverse (层序遍历)\n");
	printf("        15.InvertTree (翻转二叉树)\n");
	printf("        16.LocateParent (获取父亲结点)\n");
	printf("        17.NodeDepth (求结点的深度)\n");
	printf("        18.LocateCommonAncestor (获取最近祖先)\n");
	printf("        19.JudgeBiTree (判断二叉排序树)\n");
	printf("        0. Exit (退出菜单)\n");
	printf("----------------------------------------------\n");
	printf("    请选择你的操作[0~19]:");
	scanf("%d", &op);
	switch (op) {
	case 1:
		if (begin != 0) {
			printf("创建失败,二叉树已存在");
		} else {
			printf("输入序列为二叉树带空子树的先序
            遍历结点序列\n关键字为0时,表示空子树,为-1表示输入结束\n");
			TElemType definition[100];
			do {
				scanf("%d%s", 
                &definition[i].key, definition[i].others);
				printf("输入下一结点数据\n");
			} while (definition[i++].key != -1);
			ans = CreateBiTree(T, definition, 1);
			if (ans == OK) {
				printf("创建成功");
			} else
				printf("关键字不唯一");
			begin++;
		}
		i = 0;
		getchar();
		getchar();
		break;
	case 2:
		if (T == NULL)
			printf("销毁失败");
		else {
			ans = DestroyBiTree(T);
			if (ans == OK)
				printf("销毁成功");
			begin--;
		}
		getchar();
		getchar();
		break;
	case 3:
		if (T == NULL)
			printf("清除失败");
		else {
			ans = ClearBiTree(T);
			if (ans == OK)
				printf("清除成功");
		}
		getchar();
		getchar();
		break;
	case 4:
		ans = BiTreeEmpty(T);
		if (ans == OK)
			printf("二叉树为空");
		else
			printf("二叉树不为空");
		getchar();
		getchar();
		break;
	case 5:
		ans = BiTreeDepth(T);
		printf("二叉树深度为%d", ans);
		getchar();
		getchar();
		break;
	case 6:
		printf("请输入想要查找的关键字\n");
		scanf("%d", &e);
		getT = LocateNode(T, e);
		if (getT == NULL)
			printf("未找到");
		else {
			printf("找到了\n");
			visit(getT);
		}
		getchar();
		getchar();
		break;
	case 7:
		printf("请输入赋值的关键字,新赋值的数据\n");
		scanf("%d %d %s", &e, &value.key, value.others);
		ans = Assign(T, e, value);
		if (ans == 1)
			printf("赋值成功");
		else
			printf("赋值失败");
		getchar();
		getchar();
		break;
	case 8:
		printf("请输入关键字");
		scanf("%d", &e);
		getT = LocateBrother(T, e);
		if (getT == NULL)
			printf("无兄弟节点");
		else
			printf("已找到,且%d的兄弟节点为%d,%s",
            e, getT->data.key, getT->data.others);
		getchar();
		getchar();
		break;
	case 9:
		printf("请输入关键字、左右、插入数据\n");
		scanf("%d %d %d %s", &e, &LR, &value.key, value.others);
		ans = InsertNode(T, e, LR, value);
		if (ans == 0)
			printf("插入失败");
		else
			printf("插入成功");
		getchar();
		getchar();
		break;
	case 10:
		printf("请输入想要删除的结点的关键字\n");
		scanf("%d", &e);
		ans = DeleteNode(T, e, 1);
		if (ans == 0)
			printf("删除失败");
		else
			printf("删除成功");
		getchar();
		getchar();
		break;
	case 11:
		printf("先序遍历:");
		ans = PreOrderTraverse(T);
		getchar();
		getchar();
		break;
	case 12:
		printf("中序遍历:");
		ans = InOrderTraverse(T);
		getchar();
		getchar();
		break;
	case 13:
		printf("后序遍历:");
		ans = PostOrderTraverse(T);
		getchar();
		getchar();
		break;
	case 14:
		printf("层序遍历:");
		ans = LevelOrderTraverse(T);
		getchar();
		getchar();
		break;
	case 15:
		ans = InvertTree(T);
		printf("翻转成功");
		getchar();
		getchar();
		break;
	case 16:
		printf("输入要获取父亲结点的关键字\n");
		scanf("%d", &e);
		getT = LocateParent(T, e, 1);
		if (getT) {
			printf("%d的父亲结点是");
			visit(getT);
		} else
			printf("获取失败");
		getchar();
		getchar();
		break;
	case 17:
		printf("输入要获取深度的结点关键字\n");
		scanf("%d", &e);
	    ans = NodeDepth(T, e);
		if (ans == ERROR)
			printf("获取失败,关键字不存在");
		else
			printf("%d的深度为%d", e, ans);
		getchar();
		getchar();
		break;
	case 18:
		printf("输入要获取祖先的两个结点\n");
		scanf("%d %d", &e, &e2);
		getT = LocateCommonAncestor(T, e, e2);
		if (getT == NULL)
			printf("获取失败,关键字不存在");
		else {
			printf("最近祖先为");
			visit(getT);
		}
		getchar();
		getchar();
		break;
	case 19:
		ans = JudgeBiTree(T);
			if (ans == ERROR)
			printf("不是二叉排序树");
		else
			printf("是二叉排序树");
		k = 0; //重置静态变量,防止下一次使用该函数出现错误
		getchar();
		getchar();
		break;
	}//end of switch
	}//end of while
	printf("欢迎下次再使用本系统!\n");
	return 0;
}//end of main()
\end{lstlisting}

\section*{附录D:基于邻接表图实现的源程序}

\begin{lstlisting}
#include "stdio.h"
#include "stdlib.h"
#include <string.h>

#define TRUE 1
#define FALSE 0
#define OK 1
#define ERROR 0
#define INFEASIBLE -1
#define OVERFLOW -2
#define MAX_VERTEX_NUM 20

typedef int status;
typedef int KeyType;

//顶点类型定义
typedef struct {
	KeyType  key;
	char others[20];
} VertexType;

//表结点类型定义
typedef struct ArcNode {
	int adjvex; //顶点位置编号
	struct ArcNode  *nextarc; //下一个表结点指针
} ArcNode;

//头结点及其数组类型定义
typedef struct VNode {
	VertexType data; //顶点信息
	ArcNode *firstarc; //指向第一条弧
} VNode, AdjList[MAX_VERTEX_NUM];

//邻接表的类型定义
typedef  struct {
	AdjList vertices; //头结点数组
	int vexnum, arcnum; //顶点数、弧数
} ALGraph;

//访问顶点
void visit(VertexType v) {
	printf("%d %s  ", v.key, v.others);
}

//输出顶点信息以及连接的其他顶点的序号
void print(ALGraph G) {
	for (int i = 0; i < G.vexnum; i++) {
		ArcNode *p = G.vertices[i].firstarc;
		visit(G.vertices[i].data);
		while (p) {
			printf(" %d", p->adjvex);
			p = p->nextarc;
		}
		printf("\n");
	}
}

//根据V和VR构造图G并返回OK
status CreateCraph(ALGraph &G, VertexType V[], KeyType VR[][2]) {
	if (V[0].key == -1)
		return ERROR;//插入的点集为空
	int i, j, one = -1, two = -1, flag = 0;
	for (i = 0; VR[i][0] != -1; i++); //计算边数
	G.arcnum = i;
	for (i = 0; V[i].key != -1; i++); //计算点数
	if (i > MAX_VERTEX_NUM) //点数超过了最大的定义域时
		return ERROR;
	G.vexnum = i;
	for (i = 0; i < G.vexnum - 1; i++) 
            //判断是否有不同结点关键字相同
		for (j = i + 1; j < G.vexnum; j++)
			if (V[i].key == V[j].key)
				return ERROR;

	for (i = 0; i < G.arcnum; i++) { //判断边连接的顶点是否存在
		for (j = 0; j < G.vexnum; j++) {
			if (VR[i][0] == V[j].key)
				break;
		}
		if (j == G.vexnum) { 
            //遍历顶点集没有找到该边连接的顶点
			return ERROR;
		}
		for (j = 0; j < G.vexnum; j++) {
			if (VR[i][1] == V[j].key)
				break;
		}
		if (j == G.vexnum) {
			return ERROR;
		}
	}
	for (i = 0; i < G.vexnum; i++) { //顶点赋值给图
		G.vertices[i].data = V[i];
		G.vertices[i].firstarc = NULL;
	}
	for (i = 0; i < G.arcnum; i++) { 
                //采用头插法插入相邻边,先插入的结点后输出
		flag = 0;
		for (j = 0; j < G.vexnum; j++)
			if (VR[i][0] == G.vertices[j].data.key) { 
                //找到边的第一个顶点在顶点集中的位置
				one = j;
				break;
			}
		for (j = 0; j < G.vexnum; j++)
			if (VR[i][1] == G.vertices[j].data.key) { 
                //找到边的第二个顶点在顶点集中的位置
				two = j;
				break;
			}
		if (one == two) //环,则不插入
			continue;
		ArcNode *q = G.vertices[one].firstarc;
		while (q) {
			if (q->adjvex == two) { 
                //该结点的相邻结点已经出现过
				flag = 1;
				break;
			}
			q = q->nextarc;
		}
		if (flag == 1)
			continue;
		ArcNode *p = (ArcNode *)malloc(sizeof(ArcNode));
		p->adjvex = two;
		p->nextarc = G.vertices[one].firstarc;
		G.vertices[one].firstarc = p;
		p = (ArcNode *)malloc(sizeof(ArcNode));
		p->adjvex = one;
		p->nextarc = G.vertices[two].firstarc;
		G.vertices[two].firstarc = p;
	}
	return OK;
}

//销毁无向图G,删除G的全部顶点和边,G不存在返回ERROR,
存在并删除成功返回OK
status DestroyGraph(ALGraph &G) {
	int i;
	ArcNode *p = NULL, *q = NULL;
	for (i = 0; i < G.vexnum; i++) {
		p = G.vertices[i].firstarc;
		while (p) {
			q = p->nextarc;
			free(p);
			p = q;
		}
	}
	G.arcnum = 0;
	G.vexnum = 0;
	return OK;
}

//根据u在图G中查找顶点,查找成功返回位序,否则返回-1
int LocateVex(ALGraph G, KeyType u) {
	for (int i = 0; i < G.vexnum; i++) {
		if (u == G.vertices[i].data.key) {
			return i;
		}
	}
	return -1;
}

//根据u在图G中查找顶点,查找成功将该顶点值修改成value,返回OK;
//如果查找失败或关键字不唯一,返回ERRO
status PutVex(ALGraph &G, KeyType u, VertexType value) {
	for (int i = 0; i < G.vexnum; i++)
		if(value.key==G.vertices[i].data.key&&value.key!= u)
			return ERROR;
	int i = LocateVex(G, u);
	if (i == -1)
		return ERROR;
	G.vertices[i].data = value;
	return OK;
}

//根据u在图G中查找顶点,查找成功返回顶点u的第一邻接顶点位序,否则返回-1;
int FirstAdjVex(ALGraph G, KeyType u) {
	for (int i = 0; i < G.vexnum; i++) {
		if (u == G.vertices[i].data.key) {
			return G.vertices[i].firstarc->adjvex;
		}
	}
	return -1;
}

//v对应G的一个顶点,w对应v的邻接顶点;
//操作结果是返回v的(相对于w)下一个邻接顶点的位序;
//如果w是最后一个邻接顶点,或v、w对应顶点不存在,则返回-1。
int NextAdjVex(ALGraph G, KeyType v, KeyType w) {
	for (int i = 0; i < G.vexnum; i++) {
		if (v == G.vertices[i].data.key) {
			ArcNode *p = G.vertices[i].firstarc;
			while (p) {
				if (G.vertices[p->adjvex].data.key == w) {
					if (p->nextarc != NULL)
					return p->nextarc->adjvex;
					else
					return -1;
				}
				p = p->nextarc;
			}
		}
	}
	return -1;
}

//在图G中插入顶点v,成功返回OK,否则返回ERROR
status InsertVex(ALGraph &G, VertexType v) {
	if (G.vexnum >= MAX_VERTEX_NUM)
		return ERROR;
	for (int i = 0; i < G.vexnum; i++)
		if (v.key == G.vertices[i].data.key)
			return ERROR;
	G.vertices[G.vexnum].data = v;
	G.vertices[G.vexnum].firstarc = NULL;
	G.vexnum++;
	return OK;
}

//在图G中删除关键字v对应的顶点以及相关的弧,成功返回OK,否则返回ERROR
status DeleteVex(ALGraph &G, KeyType v) {
	int pre_vexnum = G.vexnum; //保留原始的结点个数
	int del_arc_2 = 0; //记录被删的边数
	int k, i; //记录被删结点的位置序号
	for (i = 0; i < G.vexnum; i++) { 
                //找到被删结点并删除,不会把之后结点前移
		if (G.vertices[i].data.key == v) { //找到了需要删的结点
			k = i;
			ArcNode *temp = G.vertices[i].firstarc;
			while (temp) {
			G.vertices[i].firstarc =
                    G.vertices[i].firstarc->nextarc; 
                //依次删除连接该结点的所有边
				free(temp);
				temp = G.vertices[i].firstarc;
				del_arc_2++;
			}
			break;
		}
	}
	if (i == G.vexnum) 
		return ERROR;
	for (int i = 0; i < G.vexnum; i++) {
		if (G.vertices[i].data.key == v)
			continue; //跳过删除结点
		ArcNode *p = G.vertices[i].firstarc;
                //删除其他结点中的该边
		if (p != NULL && G.vertices[p->adjvex].data.key == v) { 
			G.vertices[i].firstarc = p->nextarc;
			free(p);
			del_arc_2++;
		} else if (p)

			while (p->nextarc) {
		if (G.vertices[p->nextarc->adjvex].data.key == v) {
				ArcNode *temp = p->nextarc;
				p->nextarc = temp->nextarc;
				free(temp);
				del_arc_2++;
				temp = NULL;
			} else
				p = p->nextarc;
		}
	}

	for (int i = 0; i < G.vexnum; i++) {
		ArcNode *p = G.vertices[i].firstarc; 
		while (p) {
			if (p->adjvex > k)
				p->adjvex--;
			p = p->nextarc;
		}
	}
	for (int i = k; i < G.vexnum - 1; i++) { 
		G.vertices[i] = G.vertices[i + 1];
	}
	G.vexnum--;
	G.arcnum = G.arcnum - del_arc_2 / 2;
	if (G.vexnum == 0) //图中已经删除完顶点
		return ERROR;
	return OK;
}

//在图G中增加弧<v,w>,成功返回OK,否则返回ERROR
status InsertArc(ALGraph &G, KeyType v, KeyType w) {
	if (v == w) //插入的边是环则插入失败
		return ERROR;
	for (int i = 0; i < G.vexnum; i++)
	if (v == G.vertices[i].data.key) 
	for (int j = 0; j < G.vexnum; j++)
	if (w == G.vertices[j].data.key) { 
		ArcNode *p = G.vertices[i].firstarc;
		while (p) {
			if (p->adjvex == j) //插入的边已经存在
				return ERROR;
			p = p->nextarc;
		} //退出循环,说明插入的边不存在
		p = (ArcNode *)malloc(sizeof(ArcNode));
		p->adjvex = i;
		p->nextarc = G.vertices[j].firstarc;
		G.vertices[j].firstarc = p;
		p = (ArcNode *)malloc(sizeof(ArcNode));
		p->adjvex = j;
		p->nextarc = G.vertices[i].firstarc;
		G.vertices[i].firstarc = p;
		G.arcnum++;
		return OK;
		}
	return ERROR;
}

//在图G中删除弧<v,w>,成功返回OK,否则返回ERROR
status DeleteArc(ALGraph &G, KeyType v, KeyType w) {
	int k = 0;

	for (int i = 0; i < G.vexnum; i++) {

	if (v == G.vertices[i].data.key) { //找到了第一个顶点的位置
	for (int j = 0; j < G.vexnum; j++) {
	if (w == G.vertices[j].data.key) { //找到了第二个顶点的位置
	ArcNode *p = G.vertices[i].firstarc;
	ArcNode *p_last = p;
	while (p) {
		if (p->adjvex == j) {
			if (p == p_last) {
				G.vertices[i].firstarc = p->nextarc;
	   		free(p);
				k++;
				break;
			} else {
				p_last->nextarc = p->nextarc;
				free(p);
				k++;
				break;
			}
		}
		p_last = p;
		p = p->nextarc;
	}
	}
	}
	} else if (w == G.vertices[i].data.key) {
	for (int j = 0; j < G.vexnum; j++) {
	if (v == G.vertices[j].data.key) { //找到了第二个顶点的位置
	ArcNode *p = G.vertices[i].firstarc;
	ArcNode *p_last = p;
	while (p) {
		if (p->adjvex == j) {
			if (p == p_last) {
				G.vertices[i].firstarc = p->nextarc;
				free(p);
				k++;
				break;
				} else {
				p_last->nextarc = p->nextarc;
				free(p);
				k++;
				break;
			}
		}
		p_last = p;
		p = p->nextarc;
	}
	}
	}
	}
	}
	if (k == 0)
		return ERROR;
	G.arcnum--;
	return OK;
}

//对图G进行深度优先搜索遍历
KeyType visited[MAX_VERTEX_NUM] = {0}; //储存已经被访问过的结点
status DFSTraverse(ALGraph &G, int u) {
	if (visited[u] == 0) { 
		visit(G.vertices[u].data);
		visited[u] = 1;
	}
	ArcNode *p = G.vertices[u].firstarc;

	while (p) {

		if (visited[p->adjvex] == 0) {
			u = p->adjvex;
			DFSTraverse(G, u);
			visited[p->adjvex] = 1;
		}
		p = p->nextarc;
	}

	for (int i = 0; i < G.vexnum; i++) { 

		if (visited[i] == 0) {
			u = i;
			DFSTraverse(G, u);
		}
	}
	return OK;
}

//对图G进行广度优先搜索遍历
status BFSTraverse(ALGraph &G) {
	for (int i = 0; i < MAX_VERTEX_NUM; i++)
		visited[i] = 0;

	for (int i = 0; i < G.vexnum; i++) {

		if (visited[i] == 0) { //该结点
			visit(G.vertices[i].data);
			visited[i] = 1;
		}
		ArcNode *p = G.vertices[i].firstarc;

		while (p) {
			if (visited[p->adjvex] == 0) {
				visit(G.vertices[p->adjvex].data);
				visited[p->adjvex] = 1;
			}
			p = p->nextarc;
		}
	}
	return OK;
}

//求距离小于k的顶点集合,返回值为顶点个数,获取失败则返回ERROR
int VerticesSetLessThanK(ALGraph G, VertexType *V, KeyType v, int k) {
	int index = -1, num = 0, s = 1; 
	for (int i = 0; i < G.vexnum; i++)
		if (G.vertices[i].data.key == v)
			index = i;
	if (index == -1)
		return ERROR;
	KeyType visited[MAX_VERTEX_NUM] = {0};
	KeyType distance[MAX_VERTEX_NUM]; //记录与目标结点的距离
	for (int i = 0; i < MAX_VERTEX_NUM; i++)
		distance[i] = -1;
	visited[index] = 1;
	distance[index] = 0;
	V[num++] = G.vertices[index].data;
	while (s < k) { //求小于等于k的路径
		for (int i = 0; i < G.vexnum; i++) {
		if (distance[i] == s - 1) { 
		ArcNode *p = G.vertices[i].firstarc;
		while (p) {
			if (visited[p->adjvex] == 0) { //未被访问
			visited[p->adjvex] = 1;
			distance[p->adjvex] = s;
			V[num++] = G.vertices[p->adjvex].data;
		}
		p = p->nextarc;
		}
		}
		}
		s++;
	}
	return num;
}

//求两个顶点的最短路径,返回值为两顶点距离,采用狄克斯特拉算法
int ShortestPathLength(ALGraph G, KeyType v, KeyType w) {
	int distance[MAX_VERTEX_NUM], visited[MAX_VERTEX_NUM] = {0};
	for (int i = 0; i < MAX_VERTEX_NUM; i++)
		distance[i] = -1;
	int index = -1, s = 1, flag = 0; 
            //flag保证在一次操作中不是没有执行任务,s是当前路径长度
	for (int i = 0; i < G.vexnum; i++)
		if (G.vertices[i].data.key == v)
			index = i;
	if (index == -1) //v顶点不存在
		return ERROR;
	visited[index] = 1;
	distance[index] = 0;
	while (1) {    //求小于等于k的路径
		flag = 0;
		for (int i = 0; i < G.vexnum; i++) {
		if (distance[i] == s - 1) { 
                    //在上一轮刚被标记或者是第一轮
		ArcNode *p = G.vertices[i].firstarc;
		while (p) {
		if (G.vertices[p->adjvex].data.key == w) { 
                //找到了需要访问的另一个顶点
			return s;
		}
		if (visited[p->adjvex] == 0) { //未被访问
			visited[p->adjvex] = 1;
			distance[p->adjvex] = s;
			flag = 1; 
		}
		p = p->nextarc;
		}
		}
		}
		if (flag == 0)
			return ERROR;
		s++;
	}
}

int main( ) {
	ALGraph G;
	int op = 1;
	int j, i = 0, begin = 0, k;
	KeyType v, w, u;
	VertexType V[30], value;
	KeyType VR[100][2];
	while (op) {
	system("cls");
	printf("\n");
	printf("   Menu for Linear Table On Sequence Structure \n");
	printf("----------------------------------------------\n");
	printf("    	1. CreateCraph (创建图)\n");
	printf("        2. DestroyGraph (销毁图)\n");
	printf("    	3. LocateVex (查找顶点)\n");
	printf("        4. PutVex  (顶点赋值)\n");
	printf("    	5. FirstAdjVex (获得第一邻接顶点)\n");
	printf("        6. NextAdjVex (获得下一邻接顶点)\n");
	printf("    	7. InsertVex (插入顶点)\n");
	printf("        8. DeleteVex (删除顶点)\n");
	printf("    	9. InsertArc (插入弧)\n");
	printf("        10.DeleteArc (删除弧)\n");
	printf("    	11.DFSTraverse (深度优先遍历)\n");
	printf("        12.BFSTraverse (广度优先遍历)\n");
	printf("        13.VertexType (距离小于k的顶点集)\n");
	printf("        14.ShortestPathLength (顶点最短路径)\n");
	printf("        20.print (输出结点信息)\n");
	printf("        0. Exit (退出菜单)\n");
	printf("----------------------------------------------\n");
	printf("   请选择你的操作[0~14]:");
	scanf("%d", &op);
	switch (op) {
	case 1:
		if (begin == 1) {
			printf("创建失败,图已存在\n");
		} else {
			printf("请输入顶点数据\n");
			do {
				scanf("%d%s", &V[i].key, V[i].others);
			} while (V[i++].key != -1);
			i = 0;
			printf("请输入边的数据\n");
			do {
				scanf("%d%d", &VR[i][0], &VR[i][1]);
			} while (VR[i++][0] != -1);
			j = CreateCraph(G, V, VR);
			if (j == OK) {
				printf("创建成功");
				begin = 1;
			} else
				printf("创建失败");
		}
		getchar();
		getchar();
		break;
	case 2:
		j = DestroyGraph(G);
		if ( j == OK) {
			printf("图销毁成功!\n");
			begin = 0;
		} else
			printf("图销毁失败!\n");
		getchar();
		getchar();
		break;
	case 3:
		printf("请输入想要查找的关键字");
		scanf("%d", &u);
		j = LocateVex(G, u);
		if (j == -1)
			printf("查找失败");
		else
			printf("查找成功\n%d %s", 
                G.vertices[j].data.key, G.vertices[j].data.others);
		getchar();
		getchar();
		break;
	case 4:
		printf("输入顶点关键字并赋值\n");
		scanf("%d %d %s", &u, &value.key, value.others);
			j = PutVex(G, u, value);
		if (j == OK)
			printf("赋值成功\n");
		else
			printf("赋值失败!\n");
		getchar();
		getchar();
		break;
	case 5:
		printf("输入获取的关键字\n");
		scanf("%d", &u);
		j = FirstAdjVex(G, u);
		if ( j == -1)
			printf("查找失败\n");
		else
			printf("查找成功\n%d %s\n",
                G.vertices[j].data.key, G.vertices[j].data.others);
		getchar();
		getchar();
		break;
	case 6:
		printf("输入关键字的下一邻接点\n");
		scanf("%d %d", &v, &w);
		j = NextAdjVex(G, v, w);
		if ( j == -1)
			printf("获取失败\n");
		else
			printf("获取成功\n%d %s\n", 
                G.vertices[j].data.key, G.vertices[j].data.others);
		getchar();
		getchar();
		break;
	case 7:
		printf("输入要插入的顶点关键字\n");
		scanf("%d %s", &value.key, value.others);
		j = InsertVex(G, value);
		if (j == OK)
			printf("插入成功\n");
		else
			printf("插入失败\n");
		getchar();
		getchar();
		break;
	case 8:
		printf("输入要删除的关键字\n");
		scanf("%d", &v);
		j = DeleteVex(G, v);
		if (j == ERROR)
			printf("删除失败");
		else
			printf("删除成功");
		getchar();
		getchar();
		break;
	case 9:
		printf("输入要插入的边\n");
		scanf("%d %d", &v, &w);
		j = InsertArc(G, v, w);
		if (j == ERROR)
			printf("插入失败");
		else
			printf("插入成功");
		getchar();
		getchar();
		break;
	case 10:
		printf("输入要删除的边\n");
		scanf("%d %d", &v, &w);
		j = DeleteArc(G, v, w);
		if (j == OK)
			printf("删除成功!\n");
		else
			printf("删除失败!\n");
		getchar();
		getchar();
		break;
	case 11:
		printf("深度优先遍历:\n");
		DFSTraverse(G, 0);
		for (i = 0; i < MAX_VERTEX_NUM; i++)
			visited[i] = 0;
		getchar();
		getchar();
		break;
	case 12:
		printf("广度优先遍历:\n");
		BFSTraverse(G);
		for (i = 0; i < MAX_VERTEX_NUM; i++)
			visited[i] = 0;
		getchar();
		getchar();
		break;
	case 13:
		printf("输入顶点关键字以及距离:\n");
		scanf("%d %d", &v, &k);
		for (i = 0; i < 30; i++) {
			V[i].key = 0;
			strcpy(V[i].others, "\0");
		}
		j = VerticesSetLessThanK(G, V, v, k);
		if (j == ERROR) {
			printf("获取失败");
		} else {
			printf("以获得顶点集V\n总个数为%d\n", j);
		for (i = 0; i < j; i++) {
				visit(V[i]);
			}
		}
		getchar();
		getchar();
		break;
	case 14:
		printf("输入两个顶点关键字:\n");
		scanf("%d %d", &v, &w);
		j = ShortestPathLength(G, v, w);
		if (j == ERROR) {
			printf("获取失败");
		} else {
			printf("最短距离为%d\n", j);
		}
		getchar();
		getchar();
		break;
	case 20:
            print(G);
		getchar();
		getchar();
		break;
	}//end of switch
        }//end of while
	printf("欢迎下次再使用本系统!\n");
	return 0;
}//end of main()

\end{lstlisting}

\end{document}